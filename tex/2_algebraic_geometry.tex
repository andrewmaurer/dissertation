\chapter{Algebraic Geometry}
\label{chap:ag-git}

\section{Motivation}
\label{sec:motivation-ag}

Algebraic geometry is a remarkably powerful discipline which allows one to view the hidden geometry arising in commutative ring theory. It turns out, many of these techniques that arose from commutative ring theory hold equally well for graded-commutative rings, i.e., graded rings satisfying $a \cdot b = (-1)^{\bar a \cdot \bar b} b \cdot a$. This is handy for representation theorists because cohomology rings are very often graded-commutative.

Algebraic geometry behaves nicely when the rings in question are finitely generated over a ground field, in which case the associated geometric object is a variety.

\section{The Spectrum of a Ring}
\label{sec:spec}

\begin{definition}
  Let $\H^\bullet$ be a graded-commutative ring, which is finitely generated as a $\CC$-algebra. The \emph{spectrum} of $\H^\bullet$ is the set:
  \[
\Spec(\H^\bullet) = \left\{ [\p] \mid \p \leq \H^\bullet \text{ is a prime ideal} \right\}
  \]
  The set $\Spec(\H^\bullet)$ is equipped with a topology whose closed sets are the vanishing sets of ideals
  \[
    \Z(I) = \left\{ [\p] \in \Spec(\H^\bullet) \mid I \subseteq \p \right\}
  \]
\end{definition}

\begin{note}
  Despite the fact that $\H^\bullet$ is a graded ring, we choose to use $\Spec$ and not $\Proj$. As a result, the varieties considered are conical affine varieties and not projective varieties. This is especially useful because
  \[
    \Hom_\text{ring}(\H^\bullet_1,\H^\bullet_2) \longleftrightarrow \Mor_\text{var}(\Spec(\H^\bullet_2),\Spec(\H^\bullet_1))
  \]
  
\end{note}

In what follows, we will primarily be concerned with the topological space $\Spec(\H^\bullet)$, not with the sheaf of rings associated to it. The interested reader may consult Chapter II of Hartshorne\cite{hartshorne}.

\begin{proposition}
  Let $\H^\bullet_1$ and $\H^\bullet_2$ be commutative or graded-commutative rings. Let $I_1 \leq \H^\bullet_1$ be an ideal.
  \begin{enumerate}[(a)]
  \item A homomorphism of rings $\varphi: \H^\bullet_1 \to \H^\bullet_2$ corresponds to a morphism of varieties $\varphi^*: \Spec(\H^\bullet_2) \to \Spec(\H^\bullet_1)$. This map is determined on points by $\varphi^*([\p_2]) = \varphi^{-1}(\p_2) \leq \H^\bullet_1$.
  \item The spectrum of a quotient $\Spec(\H^\bullet_1 / I_1)$ is isomorphic to $\Z(I_1)$. Furthermore, the quotient morphism $\pi : \H^\bullet_1 \onto \H^\bullet_1 / I_1$ corresponds to the inclusion morphism $\pi^*: \Z(I) \hookrightarrow \Spec \H^\bullet_1$.
  \end{enumerate}
\end{proposition}


\section{Modules and Subvarieties}
\label{sec:mods}

As above, consider a graded-commutative ring $\H^\bullet$. Let $\M$ be a finitely-generated $\H^\bullet$-module. This module has an \emph{annihilator}
\[
  \Ann_{\H^\bullet}(\M) = \{x \in \H^\bullet \mid x . M = 0 \}
\]

\begin{definition}
  The \emph{support} of $\M$ is the vanishing set of the annihilator of $\M$:
  \[
\Supp_{\H^\bullet}(\M) = \Z\left(\Ann_{\H^\bullet}(\M)\right) \subseteq \Spec(\H^\bullet)
  \]
  
\end{definition}

\begin{proposition}
  Let $\H^\bullet$ be a commutative or graded-commutative ring, and let $\M_i$ be $\H^\bullet$-modules.
  \begin{enumerate}[(a)]
  \item If $\H^\bullet$ is graded-commutative, then $\Supp_{\H^\bullet}(\M)$ is closed and conical.
  \item $\Supp_{\H^\bullet}(\M_1 \oplus \M_2) = \Supp_{\H^\bullet}(\M_1) \cup \Supp_{\H^\bullet}(\M_2)$
  \item If $0 \to \M_1 \to \M_2 \to \M_3 \to 0$ is a short exact sequence, and $\sigma \in S_3$ is a permutation of $\{1,2,3\}$, then
    \[
      \Supp_{\H^\bullet}(\M_{\sigma(1)}) \subseteq \Supp_{\H^\bullet}(\M_{\sigma(2)}) \cup \Supp_{\H^\bullet}(\M_{\sigma(3)})
    \]
    
  \end{enumerate}
\end{proposition}

\begin{proof}
  \begin{enumerate}[(a)]
  \item This is immediate. Vanishing sets are closed by definition. Because $\Supp_{\H^\bullet}(\M)$ is the vanishing of a homogeneous ideal, it must be conical.
  \item 
  \item 
  \end{enumerate}
\end{proof}


%%% Local Variables:
%%% TeX-master: "../dissertation.tex"
%%% End: