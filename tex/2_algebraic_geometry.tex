\chapter{Ring Theoretic Notions}
\label{chap:ag-git}

\section{Main Goals}
\label{sec:ageom-goals}

The main goals of this section are to introduce the notions of depth and Krull dimension, and use them to investigate the particularly nice properties algebras have when these two quantities are equal.

\section{Ring theoretic conventions}
\label{sec:rings}

\begin{definition}
  A \emph{ring} $\Hb$ is always a $\CC$-algebra, meaning $\CC \subseteq \Hb$ and $\CC$ lies in the center of $\Hb$. A ring is \emph{commutative} if $\alpha \cdot \beta = \beta \cdot \alpha$. It is graded-commutative if it is $\ZZ$-graded and $\alpha \cdot \beta = (-1)^{\bar \alpha \bar \beta} \beta \cdot \alpha$ for homogeneous elements $\alpha, \beta$ of degrees $\bar \alpha, \bar \beta$ respectively.
\end{definition}

\begin{definition}
  The \emph{spectrum} of $\Hb$ is the set of prime ideals $\p \trianglelefteq \Hb$, equipped with the Zariski topology whose closed sets are of the form
  \[
    \Z(I) = \left\{ [\p] \in \Spec \Hb \mid I \subseteq \p \right\}
  \]
  where $I \trianglelefteq \Hb$ is some ideal.
\end{definition}

\begin{proposition} \label{prop:iso-2}
  Let $A \subseteq B$ be a pair of rings such that $B \setminus A \subseteq \mathcal{N}_B$, i.e., the complement of $A$ consists of nilpotent elements of $B$. Then \[A / \mathcal{N}_A \cong B / \mathcal{N}_B\]
\end{proposition}
\begin{proof}
  This is essentially an application of the second isomorphism theorem for rings.
\end{proof}

\begin{proposition} \label{prop:reduced}
  Let $\Hb$ be a ring, and let $\mathcal{N}$ be the ideal consisting of all nilpotent elements of $\Hb$. As a topological space, $\Spec \left(\Hb\right) \cong \Spec \left( \Hb / \mathcal{N} \right)$.
\end{proposition}
\begin{proof}
  Prime ideals of the quotient $\Hb / I$ correspond precisely to prime ideals of $\Hb$ which contain $I$. Couple this with the fact that \[\mathcal{N} = \bigcap_{[\p] \in \Spec(\Hb)} \p\] to deduce that prime ideals of $\Hb$ correspond precisely to the prime ideals of $\Hb / \mathcal{N}$.
  
\end{proof}
\begin{corollary}
  Suppose $\Hb$ is a graded-commutative ring. Let $\H^{ev} \subseteq \Hb$ be the commutative subring of all even-degree elements. As topological spaces, $\H^{ev}$ is homeomorphic to $\Hb$.
\end{corollary}
\begin{proof}
  By Proposition \ref{prop:reduced}, $\Spec(\H^{ev}) \cong \Spec(\H^{ev}/\mathcal{N}^{ev})$, and similarly $\Spec(\Hb) \cong \Spec(\Hb/\mathcal{N}^\bullet)$. Now apply Proposition \ref{prop:iso-2} to see $\H^{ev} / \mathcal{N}^{ev} \cong \Hb / \mathcal{N}^\bullet$ and trace the homeomorphisms to the desired result.
\end{proof}

What the above Corollary tells us that when studying purely topological questions involving $\Hb$, it suffices to investigate the same topological question involving an 
\section{Krull Dimension}
\label{sec:krull}

Let $\Hb$ be a ring. The $\Spec$ functor allows us to interpret $\Hb$ geometrically as the space $\Spec(\Hb)$. Neither the ring nor its spectrum has an immediately obvious notion of dimension. This can be remedied by investigating inclusions of closed sets and inclusions of ideals.

\begin{definition} Let $\Hb$ be a ring. The \emph{Krull dimension} of $\Hb$ is the largest $n$ such that there exists an inclusion of prime ideals
    \begin{equation}
      \label{eq:krull-ideals}
      \p_0 \subsetneq \p_1 \subsetneq \p_2 \subsetneq \ldots \subsetneq \p_n
    \end{equation}
    The Krull dimension of a ring is denoted $\krdim \Hb$.
\end{definition}

Chains of prime ideals, as in Equation \ref{eq:krull-ideals}, correspond to chains of irreducible subsets of $\Spec(\Hb)$. In the context of algebraic varieties, this corresponds to a point embedded in a curve embedded in a surface, and on to higher dimensions. In this way, Krull dimension agrees with geometric intuition.

\begin{example}
  \begin{enumerate}
  \item When $R$ is a ring of Krull dimension $n$, $R[x]$ is a ring of Krull dimension $n+1$.
  \item A field $k$ has Krull dimension 0, so the polynomial ring $k[x_1,\ldots,x_n]$ has dimension $n$.
  \item Let $\zeta \in k[x_1,\ldots,x_n]$ be a non-constant polynomial. The ring $k[x_1,\ldots,x_n] / \left<\zeta\right>$ has Krull dimension $n-1$.
  \item Suppose $\Hb$ is a graded-commmutative $\CC$-algebra, which is finite dimensional over $\H^0 \cong \CC$. The only prime ideal is $\H^+ = \bigoplus_{n > 0} \H^n$, so $\krdim \Hb = 0$.
  \item A principal ideal domain has dimension 1.
  \item Any Noetherian ring has finite Krull dimension.
  \end{enumerate}
\end{example}

\section{Cohen Macaulay Rings}
\label{sec:cm}

In this section we describe rings which are of especial interest to algebraic geometers. Two useful references for this section are \cite[\S 5.4]{MR1634407} and \cite[\S 18]{MR1322960}.

\begin{definition}
  Let $\Hb = \bigoplus_{n \geq 0} \H^n$ be a finitely-generated graded commutative $k$-algebra and $M = \bigoplus_{n \geq 0} M_n$ be a finitely-generated graded $A$-module. A sequence of homogeneous elements $\{\zeta_i\}_{i = 1}^r$ is a \emph{regular sequence for $M$} if for each $i$, the map
  \[
    M_n / M_n \cap \left< \zeta_1, \ldots, \zeta_{i-1}\right> \to M_{n+\bar \zeta_i} / M_{n + \bar \zeta_i} \cap \left< \zeta_1, \ldots, \zeta_{i-1}\right>
  \]
  induced by multiplication by $\zeta_i$ is injective.

  The \emph{depth} of $M$ is the length of the longest regular sequence. $M$ is \emph{Cohen-Macaulay} if $\depth(M) = \krdim(M)$.  The ring $\Hb$ is \emph{Cohen-Macaulay} if it is Cohen-Macaulay as a module over itself.
\end{definition}

\begin{example}
  The following rings are Cohen-Macaulay.
  \begin{enumerate}
  \item Any field $k$.
  \item The polynomial ring $R[x_1,\ldots,x_n]$ and the ring of formal power series $R[[x_1,\ldots,x_n]]$, where $R$ is a Cohen-Macualay ring. In particular, $k[x_1,\ldots,x_n]$ is Cohen-Macaulay.
  \item Integrally closed rings of dimension 2.
  \end{enumerate}
\end{example}

The following proposition guides intuition when it comes to recognizing regular sequences in polynomial rings. The essential idea is one would \emph{hope} the ideal $\left<\zeta_1,\ldots,\zeta_r\right>$ would define a variety of codimension $r$. If it does, this is precisely the case that the sequence $\zeta_1,\ldots,\zeta_r$ is regular.

\begin{proposition}[Macaulay]
  Suppose $I = \left<\zeta_1, \ldots , \zeta_r\right>$ is an ideal in the polynomial ring  $\CC[x_1,\ldots,x_n]$. The sequence $\{\zeta_i\}_{i=1}^r$ is regular if and only if $\krdim(A/I) = n - r$.
\end{proposition}

The following is a handy way to identify Cohen-Macaulay rings:



\begin{proposition}
  Let $\Hb$ be a finitely-generated graded-commutative $\CC$-algebra. $\Hb$ is Cohen-Macaulay if and only if there is a polynomial subring $\CC[\zeta_1,\ldots,\zeta_r] \subseteq \Hb$ generated by homogeneous elements $\zeta_i$ such that $A$ is a finitely-generated and free as a $\CC[\zeta_1,\ldots,\zeta_r]$-module.
\end{proposition}

We will be primarily concerned with the actions of reductive algebraic groups on polynomial rings. The following result of Hochster-Roberts \cite{MR0347810} will be useful in the sequel.

\begin{proposition}[Hochster-Roberts \cite{MR0347810}]
  Let $\Hb$ be a Noetherian regular $\CC$-algebra, and $\even{G}$ a reductive linear algebraic group acting acting rationally on $\Hb$. The ring of invariants $(\Hb)^{\even{G}}$ is a Cohen-Macaulay ring.
\end{proposition}

%%% Local Variables:
%%% TeX-master: "../dissertation.tex"
%%% End: