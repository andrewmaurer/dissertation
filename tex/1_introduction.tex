\chapter{Introduction}
\label{chap:intro}

\section{Motivation}

Establishing finite generation of cohomology rings is a powerful result in representation theory which links cohomology theory with commutative algebra and algebraic geometry. For example, Evens \cite{Evens-cohomology-ring} and Venkov \cite{MR0108788} each independently proved that the cohomology ring of a finite group is finitely generated. This result was used by Quillen \cite{MR0298694}, Carlson \cite{MR723070}, Chouinard \cite{MR0401943}, and Alperin-Evens \cite{MR621284} to study the cohomology variety of the finite group. This allowed those listed, among others, to use techniques from classical algebraic geometry in the study of representation theory of finite groups. Similar work has been carried out in other contexts; by Friedlander-Parshall \cite{FP-unipotent,MR860682} for restricted Lie algebras, and by Friedlander-Suslin \cite{MR1427618} for finite-dimensional cocommutative Hopf algebras.

Relative cohomology, as defined by Hochschild \cite{hochschild} is less understood than ordinary cohomology. For instance, the cohomology ring of a finite group relative to a subgroup need not be finitely generated. Indeed, Brown \cite{MR1310744} provided an example of a finite group whose relative cohomology is infinitely generated. Surprisingly, in the case $\g = \even{\g} \oplus \odd{\g}$ is a (finite-dimensional) classical Lie superalgebra, the cohomology ring of $\g$ relative to $\even{\g}$ is always finitely generated. Specifically, Boe-Kujawa-Nakano \cite{BKN-1} realized this relative cohomology ring as the invariants of a polynomial ring under the action of a reductive group. In fact, in the case of Lie superalgebras, ordinary cohomology is often times finite-dimensional as a vector space, as proved by Fuks-Leites \cite{fuks}
% \cite{MR1450424}
. This implies relative cohomology rings carry more representation theoretic information than their ordinary counterparts. Furthermore, Boe-Kujawa-Nakano \cite{BKN-1} demonstrated the atypicality of a supermodule -- a combinatorial invariant defined by Kac-Wakimoto \cite{MR1327543} -- is realized as the dimension of the support variety of that module. The geometrization of combinatorial ideas makes support variety theory useful and powerful.

One of the main results of this paper asserts that for a classical Lie superalgebra, cohomology rings relative to even subalgebras are finitely-generated over $\CC$, and the relative cohomology of a finite-dimensional module is a Noetherian module for this ring. In proving the main theorem, a spectral sequence is constructed  which relates relative Lie algebra cohomology to odd degree elements of the Lie superalgebra in an interesting way. The main theorem paves the way to define and investigate support varieties for supermodules relative to a broader class of subalgebras. The importance of this result is apparent in that cohomology relative to an even subalgebra provides a middle ground between the case of absolute cohomology of Fuks-Leites and cohomology relative to $\even{\g}$ of \cite{BKN-1}.

\section{Overview of Dissertation}
\label{sec:overview}

The majority of the original results found in this dissertation may be found in a more condensed form in the author's paper \cite{1711.02112}. This dissertation is organized as follows.

Chapter \ref{chap:ag-git} is devoted to basic ring-theoretic notions. In particular, the notions of Krull dimension and Cohen-Macaulay rings are introduced. The Krull dimension is a ring theoretic notion which corresponds to the algebro-geometric notion of dimension. Establishing finitude of Krull dimension will ensure the corresponding schemes are indeed algebraic varieties. The condition that a ring is Cohen-Macaulay corresponds to a geometric notion similar to smoothness, coupled with the ability to follow certain inductive arguments on the dimensions of varieties.

In Chapter \ref{chap:ag-git}, algebraic groups are introduced, as is their module theory and their relation to Lie algebras. There are three particularly relevant ideas in this chapter. First, the ring of invariants under a reductive group action is finitely generated. Next, representations of algebraic groups correspond to representations of their corresponding Lie algebras. Finally, a result of Hochster-Roberts \cite{MR0347810} states that the ring of invariants of a reductive group acting on a regular ring is Cohen-Macaulay.

Lie superalgebras are introduced in Chapter \ref{chap:lsa}. Particular emphasis is placed on classical Lie superalgebras, whose cohomology theory we will see is governed by geometric invariant theory. Modules are introduced, as is a parity-shift functor. Universal enveloping superalgebras are introduced to further the analogue between Lie algebras and Lie superalgebras.

In Chapter \ref{chap:rel-cohom}, relative cohomology for Lie superalgebras is introduced via a Koszul complex. Relative cohomology may be studied as in Hochschild's relative cohomology theory, as the relative derived functors of relative $\Ext_{(\g,\a)}^n(\CC,-)$ \cite{hochschild}. The product structure on Koszul cochains is investigated, and this becomes a product structure on cohomology with trivial coefficients. Important results on the cohomology are presented: a result of Boe-Kujawa-Nakano \cite{BKN-1} asserts finite generation of cohomology rings $\H^\bullet(\g,\even{\g};\CC)$ for $\g$ classical, and a result of Fuks-Leites asserts $\krdim \H^\bullet(\g,0;\CC) = 0$ in many cases. These are the extreme examples of relative cohomology rings, and the main theorem of Chapter \ref{ch:finite-generation} will establish finite generation in the intermediate cases. The chapter concludes with a realization of relative $\Ext$ groups as equivalence classes of extensions, and a computation of cohomology rings.

Chapter \ref{ch:finite-generation} contains the main theorem:
\begin{maintheorem}
  Let $\g = \even{\g} \oplus \odd{\g}$ be a classical Lie superalgebra, and $\a \leq \even{\g}$ an (even) subalgebra, and $M$ a $\g$-module.
  \begin{enumerate}[\indent\rm (a)]
    \item There is a spectral sequence $\{E_r^{p,q}\}$ which computes cohomology and satisfies
  \[
    E_2^{p,q}(M) \cong \H^p(\g,\even{\g};M) \otimes \H^q(\even{\g},\a;\CC) \Rightarrow \H^{p+q}(\g,\a;M)
  \]
  For $1 \leq r \leq \infty$, $E_r^{\bullet,\bullet}(M)$ is a module for $E_2^{\bullet,\bullet}(\CC)$. When $M$ is finite-dimensional, $E_2^{\bullet,\bullet}(M)$ is a Noetherian $E_2^{\bullet,\bullet}(\CC)$-module.
    \item Moreover, the cohomology ring $\H^\bullet(\g,\a;\CC)$ is a finitely-generated $\CC$-algebra.
  \end{enumerate}
\end{maintheorem}
This theorem is proved using a spectral sequence argument. A filtration is introduced, and pages are identified. This leads to the proof of the Main Theorem. It is proved that the edge homomorphism of the spectral sequence corresponds to restriction of functions. An example of a cohomology ring $\H^\bullet(\g,\a;\CC$ with $0 < \krdim \H^\bullet(\g,\a;\CC) < \krdim \H^\bullet(\g,\even{\g};\CC)$ is presented, which shows the utility of the theory. The final section contains results on the structure of the cohomology ring, making liberal use of the spectral sequence.

In the final chapter, relative cohomology varieties and relative support varieties for modules are introduced. We define rank varieties and conjecture these varieties are in fact equal. Natural maps of support varieties are introduced. The chapter presents some conditional results on the theory of support varieties. Finally, it is conjectured that the elusive \emph{tensor product theorem} holds, as in many analogous circumstances.



%%% Local Variables:
%%% TeX-master: "../dissertation.tex"
%%% End: