\chapter{Algebraic Groups}

\section{Overview}
\label{sec:grp-motivation}

This section's aim is to define reductive algebraic groups and introduce results important for what follows. The most relevant fact about reductive groups is Proposition \ref{prop:fg-coordinate}, which states the ring of invariants of a reductive algebraic group's action on the coordinate ring of an affine algebraic variety is finitely generated. Lie algebras are introduced as the tangent space at the identity to an affine algebraic group. Morphisms of affine algebraic groups induce morphisms of Lie algebras. This fact leads to the conclusion that group actions differentiate to Lie algebra actions.

The main reference for this chapter is Humphreys \cite{MR0396773} or Jantzen \cite{MR2015057}.

\section{Algebraic Groups ABCs}
\label{sec:basics-of-groups}

\begin{definition}
  An \emph{algebraic group} is a complex affine algebraic variety $G$ equipped with an identity element $e: \Spec(\CC) \to G$, multiplication morphism $m : G \times G \to G$ and an inverse morphism $i: G \to G$. These morphisms satisfy the typical group-theoretic axioms, expressed as commutative diagrams.

  An algebraic group is \emph{linear} if it is a subgroup of $\GL(n)$, the set of ivertible linear matrices.
\end{definition}

\begin{example} \label{ex:lag}
  The following are examples of linear algebraic groups:
  \begin{enumerate}
  \item $\GG_a$, the additive group $\CC^+$. This can be embedded into $2 \times 2$ matrices via the embedding
    \[
      a \mapsto
      \begin{pmatrix}
        1 & a \\
        0 & 1
      \end{pmatrix}
    \]
  \item The \emph{general linear group} $\GL(n)$, the set of invertible $n \times n$ matrices. When $n = 1$, we usually denote this $\GG_m = \GL(1)$, the multiplicative group $\CC^\times$.
  \item The \emph{special linear group}, $\SL(n)$, consisting of invertible matrices with determinant equal to 1.
  \item The \emph{orthogonal matrix group} $O(n)$, consisting of invertible matrices such that $M \cdot M^T = 1$. This is not connected in the Zariski topology, as the morphism $\det: O(n) \to \CC^\times$ shows. The \emph{special orthogonal group} $SO(n) = O(n) \cap \SL(n)$ is connected.
  \end{enumerate}
\end{example}

\section{Modules for Algebraic Groups}
\label{sec:group-modules}

As in the theory of finite groups, algebraic groups are best understood through their actions on vector spaces. This theory is made complete by studying the category of representations of $G$, i.e., the set of modules for $G$ with morphisms which commute with the action of $G$.

\begin{definition}
  Let $\even{G}$ be a linear algebraic group. A \emph{representation} of $\even{G}$ is a complex vector space $V$ with a morphism of algebraic varieties $\rho: \even{G} \to \GL(V)$, and elements of $G$ act on elements of $V$ via the map $g.v = \rho(g)(v)$. Alternatively, a representation may be called a \emph{$\even{G}$-module}.
\end{definition}

The following are the most important examples of representations for algebraic groups.

\begin{example}
  \begin{enumerate}
  \item The trivial representation $\even{G} \to \GG_m$ defined by $g \mapsto 1$.
  \item The standard representation of $\even{G} \subseteq \GL(n)$ acts on $\CC^n$ via $g.v = g(v)$.
  \item The determinant of a representation $\even{G} \to \GL(n) \to \GG_m$ sending $g \mapsto \rho(g) \mapsto \det(\rho(g))$.
  \item If $V$ is a representation of $\even{G}$, then so is $V^* = \Hom_\CC(V,\CC)$. This is via the action $g.f = f_g \in V^*$, where $f_g(v) = f(g^{-1}.v)$
  \item If $V$ is a representation, then so is $V^{\otimes m}$ via the diagonal action $g.(v_1 \otimes \ldots \otimes v_m) = g.v_1 \otimes g.v_2 \ldots \otimes g.v_m$. This action extends to $\even{G} \actson \T^\bullet(V)$.
  \item The symmetric and exterior products of a representation $V$ are again representations, with action inherited from the action on the tensor power.
  \end{enumerate}
\end{example}

\section{Reductive Algebraic Groups}
\label{sec:reductive}

This section introduces the notion of reductivity. Reductive algebraic groups are important because they behave particularly well with respect to invariant theory, which proves to be an invaluable tool in the computation of cohomology rings for Lie superalgebras (see Chapter~\ref{chap:rel-cohom}).

\begin{definition}
  \begin{enumerate}
  \item An element $N$ in a linear algebraic group is \emph{unipotent} if $N- I_n$ is a nilpotent matrix.
  \item The \emph{radical} of a linear algebraic group is the maximal connected, normal, solvable subgroup.
  \item The \emph{unipotent radical} of a linear algebraic group $\even{G}$ is the set of unipotent elements in the radical of $\even{G}$.
  \end{enumerate}
\end{definition}
The previous definition is a bit obscure. To see how this acts in context, refer to the following examples in which unipotent radicals are computed for common algebraic groups.
\begin{example}
  \begin{enumerate}
  \item Let $\even{G} = \GL(n)$. Up to conjugation, the radical of $G$ is the set of diagonal matrices, isomorphic to $\GG_m^n$. The only unipotent element in the radical is the identity. Thus, the unipotent radical of $\even{G}$ is $\{I_n\}$.
  \item Let $\even{G} = \SL(n)$. Up to conjugation, the radical is the set of diagonal matrices of determinant 1, isomorphic to $\GG_m^{n-1}$. The only unipotent element of this radical is the identity. The unipotent radical is again trivial.
  \item Consider the group $\GG_a$. Use the embedding of Example \ref{ex:lag} and observe that every element of $\GG_a$ is unipotent. Now notice $\GG_a$ itself is connected, normal, and solvable. Thus $\GG_a$ is its own unipotent radical.
  \end{enumerate}
\end{example}

It is a fact that in many well-behaved cases, the unipotent radical is trivial. The word for this behavior is \emph{reductive}, and proves to be of paramount importnace in representation theory and algebraic geometry.

\begin{definition}
  A linear algebraic group is \emph{reductive} if its unipotent radical is trivial. 
\end{definition}

Reductive groups have the following vitally important property, attributed to Hilbert. The main application of this theorem is to the cohomology ring of a classical Lie superalgebra relative to its even subsuperalgebra, and it turns out this ring is the invariants of a polynomial ring under a reductive group action.

\begin{proposition} \label{prop:fg-coordinate}
  Let $\even{G}$ be a reductive group acting on an affine algebraic variety $X$. The algebra of invariants $\CC[X]^{\even{G}}$ is finitely-generated over $\CC$.
\end{proposition}
\begin{corollary}
  Let $\even{G}$ be a reductive group acting on a vector space $\odd{\g}$. The ring of polynomial invariants $S^\bullet(\odd{\g}^*)^{\even{G}}$ is a finitely-generated $\CC$-algebra.
\end{corollary}

\section{The Lie Algebra of an Algebraic Group}
\label{sec:lie-algebra}

This section covers a functor $\Lie$ which maps algebraic groups to their Lie algebras. Mainly, we are interested in the way an action $\even{G} \actson V$ induces an action of $\Lie(\even{G}) = \even{\g} \actson V$.

\begin{definition}
  Let $X \subseteq \AA^n$ be an affine algebraic variety containing a point $P = \Z(\p)$, and let $\O_P = \CC[X]_\p$. The \emph{Zariski tangent space} $\T_P(X)$ is defined to be
  \begin{equation}
    \label{eq:tangent-space}
    \T_P(X) = \left(\p \O_P / \p^2 \O_P\right)^*.
  \end{equation}
  When $X = \even{G}$ is an affine algebraic group, the \emph{Lie algebra of $\even{G}$} is the tangent space at the identity,
  \begin{equation}
    \label{eq:lie-alg}
    \even{\g} = \Lie(\even{G}) = \T_1(\even{G}).
  \end{equation}
\end{definition}

\begin{proposition} \label{prop:lie-alg}
  The Lie algebra $\even{\g}$ of an affine algebraic group $\even{G}$ inherits a bilinear bracket operation $[\cdot,\cdot]: \even{\g} \otimes \even{\g} \to \even{\g}$ which satisfies the following two axioms:
  \begin{enumerate}[(L1)]
  \item $[x,x] = 0$
  \item $[x,[y,z]] = [[x,y],z] + [y,[x,z]]$
  \end{enumerate}
  Additionally, the construction of a Lie algebra is functorial, meaning a morphism $\varphi:\even{G} \to \even{G}'$ to a morphism $d\varphi:\even{\g} \to \even{\g}'$ which respects the bracket operation.
\end{proposition}

Let us introduce several important examples of Lie algebras. 

\begin{example}
  \begin{enumerate}
  \item If $\even{G}$ is an Abelian group, then $\even{\g}$ is an \emph{Abelian Lie algebra}, meaning $[x,y] = 0$ for every $x,y \in \g$.
  \item If $\even{G} = \GL(n)$, then $\even{\g} = \gl(n)$, called the \emph{general linear Lie algebra}. As a set, $\gl(n)$ is all $n \times n$ matrices with bracket given by $[A,B] = AB - BA$.
  \item If $\even{G} = \SL(n)$, then $\even{\g} = \sl(n)$, called the \emph{special linear Lie algebra}. This is the Lie subalgebra of $\gl(n)$ consisting of trace zero matrices.
  \end{enumerate}
\end{example}

When $\even{G}$ acts on a vector space $V$, this amounts to a morphism $\even{G} \to \GL(V)$. As we have seen in Proposition \ref{prop:lie-alg}, this yields a morphism of Lie algebras $\even{\g} \to \gl(V)$. This fact is noted in the following proposition.

\begin{proposition}
  The $\even{G}$-module structure on a vector space $\odd{\g}$ induces the structure of a $\Lie(\even{G}) = \even{\g}$-module structure on $\odd{\g}$.
\end{proposition}

%%% Local Variables:
%%% TeX-master: "../dissertation.tex"
%%% End: