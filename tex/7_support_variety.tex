\chapter{Support Variety Theory}
\label{ch:support}

\section{Motivation}
\label{sec:supp-motivation}

The finite-generation result of Chapter \ref{ch:finite-generation} opens the door to use the powerful machinery of algebraic geometry when studying cohomology of classical Lie superalgebras relative to an even subsuperalgebra.

In this chapter, the cohomology variety $\V_{(\g,\a)}(\CC)$ is defined, along with support varieties

$\V_{(\g,\a)}(M) \subseteq \V_{(\g,\a)}(\CC)$ for each module $M$. Natural mappings of cohomology rings yield natural mappings of cohomology varieties.

Conjectures are presented, and of particular interest is the elusive tensor-product-theorem, i.e., $\V_{(\g,\a)}(M \otimes N) = \V_{(\g,\a)}(M) \cap \V_{(\g,\a)}(N)$. The tensor product theorem has been established in several contexts in varying levels of generality. The proof always relies on concrete details, primarily through the use of explicit rank varieties. 

\section{Definition and Basic Properties}
\label{sec:def-sv}

Let $\g$ be a classical Lie superalgebra and $\a \leq \even{\g}$ a subsuperalgebra. The cohomology ring $\H^\bullet(\g,\a;\CC)$ is a graded-commutative ring, and as such the subring
\begin{equation} \label{eq:even-cohomology}
  \H^{ev}(\g,\a;\CC) = \bigoplus_{n \in \ZZ_{\geq 0}} \H^{2n}(\g,\a;\CC) \subseteq \H^\bullet(\g,\a;\CC)
\end{equation}
is a commutative, finitely-generated subring of $\H^\bullet(\g,\a;\CC)$, by Theorem \ref{thm:fg}. This leads to the first definition of this chapter.


\begin{definition}
  Let $\g = \even{\g} \oplus \odd{\g}$ be a classical Lie superalgebra with $\a \leq \even{\g}$ an even subsuperalgebra. The \emph{cohomology variety of $\g$ relative to $\a$} is the spectrum of the even subring of Equation \ref{eq:even-cohomology}:
  \[
    \V_{(\g,\a)}(\CC) = \maxspec\left( \H^{ev}(\g,\a;\CC)\right)
  \]
\end{definition}

For each $\g$-module $M$, $\Ext_{(\g,\a)}(M,M)$ is a graded module for the cohomology ring $\H^\bullet(\g,\a;\CC) = \Ext_{(\g,\a)}^\bullet(\CC,\CC)$ via the tensor product or cup product, as in Section \ref{sec:products}. Of course, $\Ext_{(\g,\a)}^\bullet(M,M)$ is a graded module for the subring $\H^{ev}(\g,\a;\CC)$. This means the annihilator
\[
  \Ann_{\H^{ev}(\g,\a;\CC)}\left(\Ext_{(\g,\a)}^\bullet(M,M)\right) \trianglelefteq \H^{ev}(\g,\a;\CC)
\]
is a homogeneous ideal for the even-degree subring of the cohomology ring.

\begin{definition}
  Let $\g = \even{\g} \oplus \odd{\g}$ be a classical Lie superalgebra with $\a \leq \even{\g}$ an even subsuperalgebra. The \emph{relative support variety} of $M$ is the vanishing set of its annihilator. In other words,
  \[
    \V_{(\g,\a)}(M) = \Z\left( \Ann_{\H^{ev}(\g,\a;\CC)}\left(\Ext_{(\g,\a)}^\bullet(M,M)\right)\right) \subseteq \V_{(\g,\a)}(\CC)
  \]
\end{definition}

Immediately, we may rephrase common properties of modules in terms of support varieties.

\begin{proposition}
  \begin{enumerate}
  \item For any $\g$-module $M$, $\V_{(\g,\a)}(M)$ is a closed, conical subvariety of $\V_{(\g,\a)}(\CC)$.
  \item For any $\g$-modules $M_1$ and $M_2$, $\V_{(\g,\a)}(M_1 \oplus M_2) = \V_{(\g,\a)}(M_1) \cup \V_{(\g,\a)}(M_2)$.
  \item Whenever $0 \to M_1 \to M_2 \to M_3 \to 0$ is a short exact sequence of $\g$-modules, and $\sigma \in \mathfrak{S}_3$ is a permutation of three letters, $\V_{(\g,\a)}(M_{\sigma(1)}) \subseteq \V_{(\g,\a)}(M_{\sigma(2)}) \cup \V_{(\g,\a)}(M_{\sigma(3)})$.
\end{enumerate}
\end{proposition}

\section{Natural Maps of Cohomology Varieties}
\label{sec:natural-maps}

In this section we exploit the realization of cohomology groups as $n$-fold extensions to see how relations between Lie superalgebras become morphisms of their associated support varieties.

Recall the realization
\begin{equation}
  \label{eq:ses-cohom}
  \H^n(\g,\a;\CC) = \{ 0 \to \CC \to E_1 \to \ldots \to E_n \to \CC \to 0 \mid \circledast \} / \sim
\end{equation}
where $\circledast$ is the condition that the sequence is exact as a sequence of $\g$-modules and splits on restriction to $\a$, and $\sim$ is an equivalence reaction obtained from the pre-equivalence relation of there existing morphisms between extensions.

\begin{definition}
  A \emph{relative subsuperalgebra} is a quadruple $(\b \leq \h, \a \leq \g)$. Here $\h \leq \g$ is a classical subsuperalgebras, in the sense that $\even{\h} \leq \even{\g}$ and $\odd{\h} \leq \odd{\g}$. Further, $\a$ is a subsuperalgebra of $\g$ and $\b$ is a subsuperalgebra of $\h$ which is also contained in $\a$. See Figure \ref{fig:relative-pair} for a pictographic definition.
\end{definition}

\begin{figure}[h]
  \centering
  \begin{tikzcd}
    \a \arrow[r,hook] & \g \\
    \b \arrow[u,hook] \arrow[r,hook] & \h \arrow[u,hook]
  \end{tikzcd}
  \caption{Relative subsuperalgebra}
  \label{fig:relative-pair}
\end{figure}




In the case that $\b \leq \h$ is a relative subsuperalgebra of the pair $\a \leq \g$, there is a natural restriction morphism of cohomology rings:
\begin{equation}
  \label{eq:nat-cohom}
  \resmap: \H^\bullet(\g,\a;\CC) \to \H^\bullet(\h,\b;\CC)
\end{equation}
This yields a natural morphism of cohomology varieties
\begin{equation}
  \label{eq:nat-supp}
  \resmap^*: \V_{(\h,\b)}(\CC) \to \V_{(\g,\a)}(\CC).
\end{equation}

There are several special cases in which the morphism of Equation \ref{eq:nat-supp} is especially useful. By Theorem \ref{thm:fg}, $\H^\bullet(\g,\a;\CC)$ is an integral extension of a quotient of $\H^\bullet(\g,\even{\g};\CC)$ via the restriction morphism (which by Proposition \ref{prop:edge} is the edge homomorphism of the spectral sequence). This means that the morphism of varieties
\[
  \resmap^* : \V_{(\g,\a)}(\CC) \to \V_{(\g,\even{\g})}(\CC)
\]
is a finite-to-one map. Further, by the results of Boe-Kujawa-Nakano \cite{BKN-1} the cohomology variety $\V_{(\g,\even{\g})}(\CC)$ may be realized as closed orbits
\[
  \V_{(\g,\even{\g})}(\CC) = \left\{ \even{G}.x \mid x \in \odd{\g} \text{ and } \even{G}.x \text{ is closed} \right\}.
\]
This proves to be an invaluable morphism, allowing us to realize elements of the support variety $\V_{(\g,\a)}(\CC)$ as closed orbits in the space $\V_{(\g,\even{\g})}(\CC)$.

\section{Rank Varieties}
\label{sec:rank-varieties}

While many common properties of support varieties follow from the general theory of modules for rings, one result that requires explicit, context-dependent computations is the proof of the elusive \emph{tensor product theorem}, stated below as a conjecture.

\begin{conjecture}[Tensor Product Theorem]
  \label{conj:tpt}
  Let $\g = \even{\g} \oplus \odd{\g}$ be a classical Lie superalgebra and $\a \leq \even{\g}$ an even subsuperalgebra. Let $M$ and $N$ be two $\g$-modules, then we may identify the support variety of their tensor product as follows:
  \[
    \V_{(\g,\a)}(M \otimes N) = \V_{(\g,\a)}(M) \cap \V_{(\g,\a)}(N).
  \]
\end{conjecture}

In many cases, the path to this theorem depends on the establishment of a rank variety description of the support variety $\V_{(\g,\a)}(M)$.

\begin{definition}
  \label{def:rank-variety}
  Let $\g = \even{\g} \oplus \odd{\g}$ be a classical Lie superalgebra and $\a \leq \even{\g}$ an even subsuperalgebra. The \emph{rank variety} of $\g$ relative to $\a$ is the variety
  \[
    \V_{(\g,\even{\g})}^\#(M) = \left\{ \even{G}.x \mid x \in \odd{\g}, \even{G}.x \text{ is closed, and } M\res_{\left<x\right>}\text{ is not projective}  \right\} \cup \{0\}
  \]
  The study of the structure of $\left<x\right>$ was conducted in Example \ref{ex:q1} and the cohomology rings were identified in Example \ref{sec:compute-H*}. 
\end{definition}

\section{Basic Results}

Let $\g$ be a classical Lie superalgebra with $\a \leq \even{\g}$ a subalgebra. We showed in Theorem \ref{thm:fg} that $\H^\bullet(\g,\a;\CC)$ is a finitely-generated graded-commutative $\CC$-algebra. Therefore, the subring of even cohomology classes $\H^{ev}(\g,\a;\CC) = \bigoplus \H^{2\bullet}(\g,\a;\CC)$ is commutative and finitely-generated over $\CC$. The \textit{cohomology variety} of $\g$ relative to $\a$ is the algebraic variety
\[
  \V_{(\g,\a)}(\CC) = \maxspec\left(\H^{ev}(\g,\a;\CC)\right).
\]
Note that since $\H^{ev}(\g,\a;\CC)$ is graded we just as well could have looked at the projectivization of $\V_{(\g,\a)}(\CC)$. When dealing with questions of connectivity it will be advantageous to use the projectivization, but in other contexts we will focus exclusively on the conical affine variety.

Recall that $\Ext_{(\g,\a)}^\bullet(M,M)$ is a module over $\H^\bullet(\g,\a;\CC)$, so its annihilator defines a subvariety called the \textit{support variety} of $M$, denoted
\[
  \V_{(\g,\a)}(M) = \Z\left({\Ann_{\H^{ev}(\g,\a;\CC)} \Ext_{(\g,\a)}^\bullet(M,M)}\right) \subseteq \V_{(\g,\a)}(\CC)
\]
where $\Z(I)$ denotes the vanishing set of $I$.

An alternative definition of the support variety is
\[
  \V_{(\g,\a)}(M) = \left\{~ \m \in \V_{(\g,\a)}(\CC) \mid \Ext_{(\g,\a)}^\bullet(M,M)_{\m} \neq 0 ~\right\}.
\]
The following are basic properties whose proof is standard and may be found in \cite{FP-unipotent}.

\begin{enumerate}
  \item For any $\g$-module $M$, $\V_{(\g,\a)}(M)$ is a closed, conical subvariety of $\V_{(\g,\a)}(\CC)$.
  \item For any $\g$-modules $M_1$ and $M_2$, $\V_{(\g,\a)}(M_1 \oplus M_2) = \V_{(\g,\a)}(M_1) \cup \V_{(\g,\a)}(M_2)$.
  \item Whenever $0 \to M_1 \to M_2 \to M_3 \to 0$ is a short exact sequence of $\g$-modules, and $\sigma \in \mathfrak{S}_3$ is a permutation of three letters, $\V_{(\g,\a)}(M_{\sigma(1)}) \subseteq \V_{(\g,\a)}(M_{\sigma(2)}) \cup \V_{(\g,\a)}(M_{\sigma(3)})$.
\end{enumerate}

In this section, we use the realization map $\Phi:\V_{(\g,\a)}(\CC) \to \V_{(\g,\even{\g})}(\CC)$ induced by restriction $\resmap: \H^\bullet(\g,\even{\g};\CC) \to \H^\bullet(\g,\a;\CC)$ to determine properties of (the image of) $\V_{(\g,\a)}(M)$. This has the advantage of taking the elusive, abstract support variety and embedding it inside of something concrete -- indeed, $\V_{(\g,\even{\g})}(\CC)$ is simply the set of closed orbits of the action $\even{G}$ on $\odd{\g}$.

% \section{Carlson supermodules}
% \label{sec:carlson-supermodules}

% Recall that the \emph{syzygy} $\Omega(M)$ of a $(\g,\a)$-module $M$ is defined to be the kernel of a relative-projective cover (in the $(\g,\a)$-category) of $M$. Further syzygies may be defined as $\Omega^0(M) = M$ and $\Omega^{n+1}(M) = \Omega(\Omega^n(M))$. An element $\zeta \in \H^n(\g,\a;M)$, is represented by a morphism $\Omega^n(M) \to \CC$, and its \emph{Carlson supermodule} is defined by $L_\zeta = \ker(\zeta: \Omega^n(M) \to \CC)$.

% In the case that $\a = \even{\g}$, the projectives are finite-dimensional. For arbitrary $\a$, this may not be the case, as the $(\g,\a)$-category may fail to be self-injective. As such, we must be very careful when dealing with (possibly infinite-dimensional) Carlson modules.

% % In the case that $L_\zeta$ is finite-dimensional, we have the following important dictionary between hypersurfaces and modules.

% % \begin{proposition}
% %   Let $\g = \even{\g} \oplus \odd{\g}$ be a Lie superalgebra and $\a \leq \even{\g}$ a subalgebra. Suppose the map $\Phi:\V_{(\g,\a)}(\CC) \to \V_{(\g,\even{\g})}(\CC)$ is a closed embedding. Let $\zeta \in \H^n(\g,\a;\CC)$. Then $\V_{(\g,\a)}(L_\zeta)$ is the hypersurface defined by $\zeta$. We denote this hypersurface $\V_{(\g,\a)}\left<\zeta\right>$.
% % \end{proposition}
% % \begin{proof}
% %   $\Phi(\V_{(\g,\a)}(L_\zeta))$ is a closed subvariety of $\V_{(\g,\even{\g})}(\CC)$, and as such is the intersection of hypersurfaces necessarily of the form $\V_{(\g,\even{\g})}(L_{\hat \zeta_i})$, where $\hat \zeta_i \in \H^{n_i}(\g,\even{\g};\CC)$.
% % \end{proof}

% \begin{proposition}
%   Let $\g = \even{\g} \oplus \odd{\g}$ be a classical Lie superalgebra, $\a \leq \even{\g}$ a subalgebra, and $M$ a finite-dimensional $\g$-module such that $\Omega(M)$ and $\Omega^{-1}(M)$ are both finite-dimensional. The following equalities hold:
%   \[
%     \V_{(\g,\a)}(M) = \V_{(\g,\a)}(M^*) = \V_{(\g,\a)}(\Omega(M)) = \V_{(\g,\a)}(\Omega^{-1}(M)).
%   \]

% \end{proposition}

% \begin{proposition}
%   If $\zeta_1 \in \H^r(\g,\a;\CC)$ and $\zeta_2 \in \H^s(\g,\a;\CC)$, then there is a short exact sequence
%   \[
%     0 \to \Omega^r L_{\zeta_2} \to L_{\zeta_1 \cdot \zeta_2} \oplus P \to L_{\zeta_1} \to 0.
%   \]

% \end{proposition}

% Given an element $\zeta \in \H^n(\g,\even{\g};\CC)$ and its Carlson module $L_\zeta$, there are two natural ways to obtain a $(\g,\a)$-module. Namely, via restriction on the module level $L_\zeta\res_{(\g,\a)}$ or the restriction on cohomology, $L_{\resmap \zeta}$.

% \begin{proposition}
%   Let $\g = \even{\g} \oplus \odd{\g}$ be a classical Lie superalgebra, $\a \leq \even{\g}$ a subalgebra, and $\zeta \in \H^n(\g,\even{\g};\CC)$ a homogeneous element of cohomology. Let $\bar \zeta = \resmap \zeta \in \H^n(\g,\a;\CC)$. Then
%   \[
%     L_\zeta\res_{(\g,\a)} \cong L_{\bar \zeta}.
%   \]
% \end{proposition}
% \begin{proof}
%   Relative projective modules for $(\g,\a)$ are projective modules for $(\g,\even{\g})$, since relative projectives for $\even{\g}$ are simply summands of $U(\g) \otimes_{U(\a)} L \cong U(\g) \otimes_{U(\even{\g})} U(\even{\g}) \otimes_{U(\a)} \otimes L$. As such, we may take relative projective covers and relative syzygies in the $(\g,\a)$-category to obtain $\Omega^n_{(\g,\a)}(\CC) \cong \Omega^n_{(\g,\even{\g})}(\CC)$. From this, it is clear that $L_{\bar \zeta}$ appears by restriction, and thus the isomorphism holds.
% \end{proof}
% % END DELETED PART

\section{Realizability}
\label{sec:realizability}

In this section we address the question of realizability, initially studied by Carlson \cite{MR723070}. As we are using results of Bagci-Kujawa-Nakano \cite{MR2448087}, we need additional assumptions on the Lie superalgebra $\g$, namely we require the superalgebra is \emph{stable} and \emph{polar} in addition to being classical. These assumptions originate in geometric invariant theory, and hold for $\gl(m|n)$ -- see \cite[\S 3.2-3.3]{BKN-1} for a thorough description.

\begin{definition}
  Let $\g = \even{\g} \oplus \odd{\g}$ be a classical, stable, and polar Lie superalgebra with $\a \leq \even{\g}$ a subalgebra. We say $(\g,\even{\g})$-module $M$ is \emph{natural} (with respect to $\a$) if $\V_{(\g,\even{\g})}(M) \cap \Phi\left(\V_{(\g,\a)}(\CC)\right) = \Phi\left(\V_{(\g,\a)}(M)\right)$. The subalgebra $\a$ is \emph{natural} if every $\g$-module is natural with respect to $\a$.
\end{definition}

The paper of Bagci-Kujawa-Nakano \cite[Theorem 8.8.1]{MR2448087} demonstrated that every closed conical subvariety of $\V_{(\g,\even{\g})}(\CC)$ is realized as the support variety of a $(\g,\even{\g})$-module.

\begin{proposition}
  Let $\g = \even{\g} \oplus \odd{\g}$ be a classical, stable, and polar Lie superalgebra with $\a \leq \even{\g}$ a natural subalgebra. Let $X \subseteq \V_{(\g,\a)}(\CC)$ be a closed, conical subvariety. There exists a $(\g,\a)$-module $M$ such that $\Phi\left(\V_{(\g,\a)}(M)\right) = \Phi(X)$.
\end{proposition}
\begin{proof}
  The realization theorem holds for $(\g,\even{\g})$-modules, so choose $M$ such that $\V_{(\g,\even{\g})}(M) = \Phi(X)$. By naturality, $\Phi(\V_{(\g,\a)}(M)) = \Phi(\V_{(\g,\a)}(\CC)) \cap \V_{(\g,\even{\g})}(M) = \Phi(X)$.
\end{proof}

\section{Tensor products}
\label{sec:tensor-products}

A tensor product theorem gives us the ability to geometrically control the support theory of tensor products of modules. Historically, this has been a very elusive property of support varieties, often times requiring support varieties recognized in some other way. For example, in the case of finite groups, the tensor product theorem was not shown until support varieties were determined to be isomorphic to the very concrete rank varieties \cite{MR621284}.

In this section, we circumvent this issue by considering only superalgebras which satisfy the tensor product theorem relative to $(\g,\even{\g})$, and using the realization map to intersect supports of $(\g,\a)$-modules inside $\V_{(\g,\even{\g})}(\CC)$.

\begin{definition}
  Let $\g = \even{\g} \oplus \odd{\g}$ be a Lie superalgebra with subalgebra $\a \leq \even{\g}$. The pair $(\g,\a)$ is said to satisfy the \emph{tensor product theorem} if $\V_{(\g,\a)}(M \otimes N) = \V_{(\g,\a)}(M) \cap \V_{(\g,\a)}(N)$ for all modules $M,N$.
\end{definition}

Lehrer-Nakano-Zhang proved the tensor product theorem hold for the pair $(\gl(m|n),\even{\gl(m|n)})$, \cite[Theorem 5.2.1]{MR2836115}

\begin{proposition}
  Let $\g = \even{\g} \oplus \odd{\g}$ be a Lie superalgebra which satisfies the tensor product theorem relative to $\even{\g}$, and $\a \leq \even{\g}$ a natural subalgebra of $\g$.  Then $\Phi(\V_{(\g,\a)}(M \otimes N)) = \Phi(\V_{(\g,\a)}(M)) \cap \Phi(\V_{(\g,\a)}(N))$.
\end{proposition}
\begin{proof}
  One has:
  \begin{align*}
    \Phi(\V_{(\g,\a)}(M \otimes N)) &= \Phi(\V_{(\g,\a)}(\CC)) \cap \V_{(\g,\even{\g})}(M \otimes N) \\
                                   &= \left(\Phi(\V_{(\g,\a)}(\CC)) \cap \V_{(\g,\even{\g})}(M)\right) \cap \left( \Phi(\V_{(\g,\a)}(\CC)) \cap \V_{(\g,\even{\g})}(N)\right) \\
    &= \Phi(\V_{(\g,\a)}(M)) \cap \Phi(\V_{(\g,\a)}(N)).
  \end{align*}
\end{proof}

% BEGIN DELETED PART
\begin{proposition}
  Let $\g = \even{\g} \oplus \odd{\g}$ be a classical Lie superalgebra with $\a \leq \even{\g}$ a submodule. Denote by $\Phi: \V_{(\g,\a)}(\CC) \to \V_{(\g,\even{\g})}(\CC)$ the restriction morphism. If $M$ is a $(\g,\even{\g})$-module, then
  \[
    \V_{(\g,\even{\g})}(M) \cap \Phi\left(\V_{(\g,\a)}(\CC)\right) = \Phi\left(\V_{(\g,\a)}(M)\right).
  \]

\end{proposition}
\begin{proof}
  First, suppose $\m \in \Phi\left(\V_{(\g,\a)}(M)\right)$. This means there is a maximal ideal $\widetilde \m \trianglelefteq \H^\bullet(\g,\a;\CC)$ such that  $\resmap^{-1}(\widetilde \m) = \m$ and $\widetilde{\m} \supseteq \Ann_{\H^\bullet(\g,\a;\CC)}\Ext_{(\g,\a)}^\bullet(M,M)$. Of course, $\m \in \Phi\left(\V_{(\g,\even{\g};\CC)}(\CC)\right)$, so it suffices to show $\m \in \V_{(\g,\even{\g})}(M)$, i.e., that every element $\zeta \in \H^\bullet(\g,\even{\g};\CC)$ which annihilates $\Ext_{(\g,\even{\g})}^\bullet(M,M)$ is an element of $\m$. Consider such a $\zeta$ and an extension $0 \to M \to \ldots \to M \to 0$. We wish to show $\resmap \zeta$ annihilates $\Ext_{(\g,\a)}^\bullet(M,M)$ -- it is only clear that $\resmap \zeta$ annihilates $\resmap \left( \Ext_{(\g,\even{\g})}^\bullet (M,M)\right)$, i.e., it may not annihilate some sequences which split on restriction to $\a$ but not on restriction to $\even{\g}$.

  Coversely, consider an element $\m \in \V_{(\g,\even{\g})}(M) \cap \Phi\left(\V_{(\g,\a)}(\CC)\right)$. Choose $\widetilde{\m} \trianglelefteq \H^\bullet(\g,\a;\CC)$ such that $\m = \resmap^{-1}(\widetilde{\m})$. We wish to show $ \widetilde{\m} \supseteq \Ann_{\H^\bullet(\g,\a;\CC)}\Ext^\bullet_{(\g,\a)}(M,M)$. Choose $\zeta \in \Ann_{\H^\bullet(\g,\a;\CC)}\Ext_{(\g,\a)}^\bullet(M,M)$. Because the restriction homomorphism is surjective (on the level of rings), there exists an element $\hat \zeta \in \H^\bullet(\g,\even{\g};\CC)$ which restricts to $\zeta$. Such a $\hat \zeta$ annihilates all elements of $\Ext_{(\g,\even{\g})}^\bullet(M,M)$, since the action is defined by tensor product (over $\CC$) of exact sequences. Thus $\hat \zeta \in \Ann_{\H^\bullet(\g,\even{\g};\CC)}\Ext_{(\g,\even{\g})}^\bullet(M,M) \subseteq \m$, and by construction $\resmap \hat \zeta = \zeta \in \widetilde{\m}$.
\end{proof}
% END DELETED PART


\begin{corollary}
  Suppose $\V_{(\g,\a)}(\CC) \to \V_{(\g,\even{\g})}(\CC)$ is a closed embedding. If a variety $X \subseteq \V_{(\g,\even{\g})}(\CC)$ is realized by a  $(\g,\even{\g})$-module, then $X \cap \V_{(\g,\a)}(\CC)$ is realized by a $(\g,\a)$-module.
\end{corollary}

\section{Connectedness of support varieties}
\label{sec:connectivity}

This section investigates connectedness of support varieties, motivated by Benson's presentation \cite{MR1634407}.

\begin{proposition}
  Let $\g = \even{\g} \oplus \odd{\g}$ be a classical, stable, and polar Lie superalgebra with $\a \leq \even{\g}$ a natural subalgebra. Suppose $\Phi(\V_{(\g,\a)}(M)) = X \cup Y$ with $X \cap Y = \{0\}$. Then there exist modules $M_1$ and $M_2$ such that $M = M_1 \oplus M_2$, $X = \Phi(\V_{(\g,\a)}(M_1))$, $Y = \Phi(\V_{(\g,\a)}(M_2))$, and \[
    \Phi(\V_{(\g,\a)}(M)) = \Phi(\V_{(\g,\a)}(M_1)) \cup \Phi(\V_{(\g,\a)}(M_2)).
  \]
\end{proposition}
\begin{proof}
  By realizability for $(\g,\even{\g})$, because $\Phi(\V_{(\g,\a)}(M))$ is a closed conical subvariety of $\V_{(\g,\even{\g})}(M)$, there exist $M_1$ and $M_2$ such that $\Phi(\V_{(\g,\a)}(M)) = \V_{(\g,\even{\g})}(M_1) \cup \V_{(\g,\a)}(M_2)$. Using this fact, we may compute:
  \begin{align*}
    \Phi(\V_{(\g,\a)}(M)) &= \V_{(\g,\even{\g})}(M_1) \cup \V_{(\g,\even{\g})}(M_2) \\
                         &= \left(\Phi(\V_{(\g,\a)}(\CC)) \cap \V_{(\g,\even{\g})}(M_1)\right) \cup \left(\Phi(\V_{(\g,\a)}(\CC)) \cap \V_{(\g,\even{\g})}(M_2)\right) \\
    &= \Phi(\V_{(\g,\a)}(M_1)) \cup \Phi(\V_{(\g,\a)}(M_2)).
  \end{align*}
\end{proof}


% \begin{proposition}
%   Let $\g = \even{\g} \oplus \odd{\g}$ be a classical Lie superalgebra, $\a \leq \even{\g}$ an even subalgebra, and $M$ a finite-dimensional $(\g,\a)$-module. If $\V_{(\g,\a)}(M) = \V_1 \cup \V_2$ with $\V_1 \cap \V_2 = \{0\}$, then $M \cong M_1 \oplus M_2$ with $\V_i = \V_{(\g,\a)}(M_i)$.
% \end{proposition}
% \begin{proof}
%   The proof of Benson's 5.12.1 carries over verbatim with the exception of requiring a tensor product theorem and the fact $\V_{(\g,\a)}(L_\zeta) = \V_{(\g,\a)}\left<\zeta\right>$. The first will hold for $\gl(m|n)$, and more generally for any superalgebra which satisfies the tensor product theorem for $(\g,\even{\g})$, whereas the second will hold once I work out the highlighted realizability kink above.
% \end{proof}


%%% Local Variables:
%%% TeX-master: "../dissertation.tex"
%%% End: