\chapter{Lie Superalgebras}
\label{chap:lsa}

\section{Motivation}
\label{sec:motivate-lsa}

A \emph{Lie superalgebra} is a $\ZZ_2$-graded analogue of a Lie algebra. Lie superalgebras originated in the physical theory of \emph{supersymmetry} and play a similar role as Lie algebras, in that they arise as tangent spaces to \emph{Lie supergroups} at the identity element.

A thorough overview of Lie superalgebra theory is provided by Victor Kac \cite{Kac}. A main result of this paper is the classification of simple classical Lie superalgebras.

\begin{theorem}[Kac, \cite{Kac}]
  A simple classical Lie superalgebra is isomorphic to either to one of the simple Lie algebras $A_n$, $B_n$, \ldots, $E_8$ or to one of $A(m,n)$, $B(m,n)$, $C(n)$, $D(m,n)$, $D(2,1;\alpha)$, $F(4)$, $G(3)$, $P(n)$, or $Q(n)$.
\end{theorem}

\section{Definition and Examples}

We start this chapter with a definition.

\begin{definition}
  A \emph{superspace} is a $\ZZ_2$-graded complex vector space $V = \even{V} \oplus \odd{V}$. An element of $V_i$ is called \emph{homogenous of degree $i$}. The \emph{superdimension} of $V$ is the ordered pair $\sdim V = (\dim \even{V}, \dim \odd{V})$.
\end{definition}

It turns out that the vector space of homomorphisms also has the natural structure of a superspace.

\begin{example} \label{ex:hom}
  Let $V$ and $W$ be superspaces. The vector space of linear homomorphisms $\Hom_\CC(V,W)$ is naturally a superspace:
  \begin{align*}
    \even{\Hom_\CC(V,W)} &= \{ \varphi \mid \varphi(V_i) \subseteq \varphi(W_i)\} \\
    \odd{\Hom_\CC(V,W)} &= \{ \varphi \mid \varphi(V_i) \subseteq \varphi(W_{i + \bar 1})\}
  \end{align*}
  In this way,
  \[
    \Hom_\CC(V,W) = \even{\Hom_\CC(V,W)} \oplus \odd{\Hom_\CC(V,W)}
  \]
  
\end{example}

With the notion of superspace defined, we may now define the concept of a Lie superalgebra. This is an algebraic object which, in the theory of supersymmetry \cite{varadarajan}, plays the role analogous to that of a Lie algebra in representation theory of algebraic groups. In supersymmetry, formulas are typically defined on homogeneous elements and extended by linearity. Additionally, commutation of two homogeneous quantitites results in an additional factor of $(-1)$ to the product of their degrees.

\begin{definition}
  A \emph{Lie superalgebra} is a superspace $\g = \even{\g} \oplus \odd{\g}$, equipped with a bilinear bracket $[\cdot,\cdot]: \g \otimes \g \to \g$ satisfying the following two properties:
  \begin{enumerate}
  \item [(S1)] For $x,y,z$ homogeneous elements of $\g$,
    \[
      [x,[y,z]] = [[x,y],z] + (-1)^{\bar x \cdot \bar y} [y, [x,z]]
    \]
    
  \item [(S2)] For $x,y$ homogeneous elements of $\g$,
    \[
      [x,y] + (-1)^{\bar x \cdot \bar y} [y,x] = 0
    \]
    
  \end{enumerate}
\end{definition}

It is worth noting that the even subsuperalgebra $\even{\g}$ is, in fact, a Lie algebra. Furthermore, the subset of odd elements $\odd{\g}$ is a module for the Lie algebra $\even{\g}$.

\filbreak
\begin{example}~
  \begin{enumerate}
    \item Let $V = \even{V} \oplus \odd{V}$ be a superspace of superdimension $(m|n)$. The \emph{general linear Lie superalgebra} $\gl(V)$ or $\gl(m|n)$ is the superspace $\Hom_\CC(V,V)$, with grading of Example \ref{ex:hom}, visualized as
  \[
    \even{\gl(m|n)} =
    \left(
      \begin{array}{c|c}
        A_{m \times m} & 0 \\
        \hline
        0 & A_{n \times n}
      \end{array}
    \right)
    \text{ and }
    \odd{\gl(m|n)} =
    \left(
      \begin{array}{c|c}
        0 & A_{m \times n} \\
        \hline
        A_{n \times m} & 0
      \end{array}
    \right)
  \]
  The bracket operation on $\gl(m|n)$ is defined for homogeneous elements via
  \begin{equation} \label{eq:gl-bracket}
    [A,B] = A \cdot B - (-1)^{\bar A \cdot \bar B} B \cdot A
  \end{equation}
  \item Consider the matrix $A \in \gl(m|n)$, decomposed as
  \[
    A = 
    \left(
      \begin{array}{c|c}
        A_{m \times m} & A_{m \times n} \\
        \hline
        A_{n \times m} & A_{n \times n}
      \end{array}
    \right)
  \]
  The \emph{supertrace of $A$} is $\sTr(A) = \Tr(A_{m \times m}) - \Tr(A_{n \times n})$. The \emph{special linear Lie superalgebra} is $\sl(V)$ or $\sl(m|n)$ and consists of all matrices in $\gl(V)$ with supertrace 0, i.e.,
  \[
    \sl(m|n) = \left\{ A \in \gl(m|n) \mid \sTr(A) = 0 \right\}
  \]
\end{enumerate}
\end{example}

\begin{example}
  \label{ex:q1}
  Let $\g$ be a Lie superalgebra. We will classify all subsuperalgebras generated by a single homogeneous element $x \in \g$.
  \begin{enumerate}
  \item If $x \in \even{\g}$, then $[x,x] = 0$. As such $\left<x\right>$ is a one-dimensional simple Lie algebra $\g \cong \CC \oplus \{0\}$
  \item If $x \in \odd{\g}$, and $[x,x] = 0$ then there are no even elements and thus $\g \cong 0 \oplus \CC$.
  \item \label{ex:q13} If $x \in \odd{\g}$ and $[x,x] = y \neq 0$, then the super Jacobi axiom says $[x,y] = [x,[x,x]] = [[x,x],x] - [x,[x,x]]$. Applying super anticommutativity yields $[x,y] = 0$. The multiplication table for this Lie superalgebra is presented in Figure \ref{fig:q1-mult}. Lie superalgebras isomorphic to this one are referred to as \emph{of type $\mathbf{q}(1)$}.
    \begin{figure}[h]
      \label{fig:q1-mult}
      \centering
      \begin{tabular}{r|| c | c |}
$[\cdot,\cdot]$ & y & x \\
\hline\hline
        y & x & 0 \\
        x & 0 & 0 \\
\hline
\end{tabular}
      \caption{Multiplication table for $\mathbf{q}(1)$}
    \end{figure}
  \end{enumerate}
\end{example}

\section{Classical Lie Superalgebras}
\label{sec:classical-superalgebras}

This section introduces a broad class of Lie superalgebras whose structure is governed by the theory of reductive algebraic groups. Later, we will see that the cohomology theory of these Lie superalgebras is also determined by the invariant theory, which behaves particularly nicely. 

\begin{definition}
  A \emph{classical Lie superalgebra} is a Lie superalgebra $\g = \even{\g} \oplus \odd{\g}$ such that there exists a reductive algebraic group $\even{G}$ which acts on $\odd{\g}$ which satisfies
  \begin{enumerate}
  \item $\even{\g} = \Lie(\even{G})$
  \item The action of $\even{G} \actson \odd{\g}$ differentiates to yield the adjoint action $\even{\g} \actson \odd{\g}$.
  \end{enumerate}
\end{definition}

\begin{example}[Lie superalgebra of type $\q(n)$, as in {\cite[\S 8.3]{BKN-1}} ] \label{ex:q(n)}
  We define a Lie superalgebra called $\q(n)$ as a Lie subsuperalgebra of the special linear Lie superalgebra $\q (n) \leq \gl(n \mid n)$.
  \[
    \q(n) =
    \left\{
    \left(
    \left.    
      \begin{array}{c|c}
        A & B \\
        \hline
        B & A
      \end{array}
    \right)
    \right| A,B \in \M_{n \times n}(\CC)
    \right\}
  \]
  A quick computation shows that $\sdim \q(n) = (n^2 \mid n^2)$ (and therefore $\dim_\CC \q(n) = 2n^2$), $\even{\q(n)} \cong \gl(n)$, $\odd{\q(n)} \cong \gl(n)$, and $\odd{\g}$ is the adjoint representation of $\even{\g}$. In this way, $\q(n)$ is a classical Lie superalgebra with $\even{G} = \GL(n)$, and $\even{G} \actson \odd{\g}$ via conjugation, yielding the adjoint action of $\even{\g} \actson \odd{\g}$.

  Additionally, we may verify the Lie superalgebra of Example \ref{ex:q1} Part \ref{ex:q13} is indeed the classical Lie superalgebra described above. This follows by taking a basis of the form
  \[
    x = 
        \left(
      \begin{array}{c|c}
        0 & 1 \\
        \hline
        1 & 0
      \end{array}
    \right)
    \text{ and }
    y = 
        \left(
      \begin{array}{c|c}
        1 & 0 \\
        \hline
        0 & 1
      \end{array}
    \right)
  \]
  and verifying that the multiplication table of Figure \ref{fig:q1-mult} is valid.
\end{example}


\section{Modules for Lie Superalgebras}
\label{sec:modules}

As with any object in abstract algebra, we care not simply about Lie superalgebras on their own, but also about their actions on vector spaces. Because of the grading on $\U(\g)$ (introduced in Section \ref{sec:universal-enveloping-superalgebras}), we require $\g$-modules to be graded $M = \even{M} \oplus \odd{M}$. With this requirement, the category of $\g$-modules is no longer Abelian. In order to make use of the tools of homological algebra, we make use of the subcategory whose objects are $\g$-modules and whose morphisms are \emph{even} homomorphisms of $\g$-modules. This subcategory is useful when the \emph{parity change functor} $\Pi$ is used, in which case all data contained in the catogory of $\g$-modules may be recovered.

\begin{definition}
  A module may be defined in the following three equivalent ways, each of which is useful in certain cases.
  \begin{enumerate}
  \item $V = \even{V} \oplus \odd{V}$ is a graded module for the universal enveloping superalgebra $\U(\g)$ (to be defined in \ref{sec:universal-enveloping-superalgebras}).
  \item $V = \even{V} \oplus \odd{V}$ is a graded complex vector space and $\rho : \g \to \gl(V)$ is an even homomorphism of vector spaces. The action is $g.v = \rho(g)(v)$.
  \item $V = \even{V} \oplus \odd{V}$ is a graded complex vector space and $\g$ acts on $V$ in a linear fashion such that the following condition holds:
\begin{equation}
  g_1.(g_2.x) - (-1)^{\bar g_1 \bar g_2} g_2. (g_1.x) = [g_1,g_2].x
\end{equation}
for all homogeneous $g_1, g_2 \in \g$.
  \end{enumerate}
\end{definition}

\begin{definition}
  A \emph{homomorphism} of $\g$-modules $f: M \to N$ is a homogeneous linear map (i.e., $f \in \even{\Hom(M,N)} \cup \odd{\Hom(M,N)}$) satisfying the following property:
  \[
    f(g.m) = (-1)^{\bar f \cdot \bar g} g.f(m)
  \]
  for $g \in \g$, $m \in M$.
\end{definition}

Unfortunately, the category of $\g$-modules is not an Abelian category. We remedy this situation by considering the even subcategory, whose objects are $\g$-modules and whose morphisms are even homomorphisms $\even{\Hom_\g(M,N)}$.

\begin{proposition}
  The category $\even{\Mod(\g)}$ is an Abelian category.
\end{proposition}

\begin{definition}
  The \emph{parity change functor} is a functor $\Pi: \Mod(\g) \to \Mod(\g)$ which switches the grading of modules. Symbolically $\even{\Pi(M)} = \odd{M}$, and $\odd{\Pi(M)} = \even{M}$.
\end{definition}

\begin{proposition}
  Let $V = \even{V} \oplus \odd{V}$ be a superspace. Then $\gl(V)$ is naturally isomorphic to $\gl(\Pi(V))$. This isomorphism may be visualized as follows:
\begin{equation}
    \left(
      \begin{array}{c|c}
        A_{n \times n} & B_{n \times m} \\
        \hline
        C_{m \times n} & D_{m \times m}
      \end{array}
    \right)
    \mapsto
    \left(
      \begin{array}{c|c}
        D_{m \times m} & C_{m \times n} \\
        \hline
        B_{n \times m} & A_{n \times n}
      \end{array}
    \right)
\end{equation}
  
  \end{proposition}

The above proposition allows us to glean all information from $\Mod(\g)$ from $\even{\Mod(\g)}$ in the following way.
\begin{corollary}
  If $M$ is a $\g$-module, $\Pi(M)$ is a $\g$-module. Further, an odd homomorphism $M \to N$ is simply an even homomorphism $M \to \Pi(N)$.
\end{corollary}


\section{Universal Enveloping Superalgebras}
\label{sec:universal-enveloping-superalgebras}

When studying representations of an algebraic object $G$, it is useful to find a ring $R$ whose modules correspond precisely to $G$-representations. This section is devoted to constructing the universal enveloping superalgebra $\U_s(\g)$ associated to a Lie superalgebra, such that $\g$-modules 
\begin{definition}
  For an associative superalgebra $A$, denote by $\Lie(A)$ the vector space $A$ with bracket operation given by the commutator of Equation \ref{eq:gl-bracket}.

  
  The \emph{universal enveloping superalgebra} of a Lie superalgebra $\g$ is an associative superalgebra $\U_s(\g)$ equipped with a morphism $i: \g \to \U_s(\g)$ such that given any other superalgebra $A$ with a morphism $j: \g \to \Lie(A)$ there exists a unique homomorphism $\theta: \U_s(\g) \to A$ such that $j = \theta \circ i$.

  Explicitly, a universal enveloping superalgebra may be obtained as a quotient of the tensor superalgebra\footnote{Simply the tensor algebra, with grading remembered.} by the ideal generated by elements of the form $[x,y] - x \otimes y (-1)^{\bar x \bar y} y \otimes x$.
\end{definition}

\begin{proposition}
  The following categories are equivalent:
  \begin{enumerate}
  \item The category of graded $\U_s(\g)$-modules (in the sense of ring theory).
  \item The category of $\g$-modules (in the sense of Section \ref{sec:modules}).
\end{enumerate}
\end{proposition}

%%% Local Variables:
%%% TeX-master: "../dissertation.tex"
%%% End: