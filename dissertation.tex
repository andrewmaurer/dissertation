% File 'ugasample.tex' -- Michael A. Covington, Isidor Ruderfer (Revised 2007 April 11)
% This is an example of how to format a thesis or dissertation
% using LaTeX 2e with 'uga.sty' VERSION 3.2 OR LATER.



\documentclass[12pt]{report}

% If there are any other \usepackage commands, put them here

\usepackage{uga}
\usepackage{macros}
% \usepackage{draftwatermark}
% \SetWatermarkScale{4}
%\usepackage{showkeys}

\newcommand{\TODO}[2]{\todo[inline,color=yellow]{\textbf{#1}: {#2}}}

% DO NOT put any more \usepackage commands after \usepackage{uga}, except:

% The following was recommended in the past so you can have hyperlinks in the eventual PDF output.
% It will produce profuse error messages when you have any unusual symbols or italics
% in a section title, or when you use pseudochapters.  Just ignore them.
% To activate it, remove the initial %.  In case of problems, put the % back in.
%\usepackage[ps2pdf, bookmarks=true, breaklinks=true]{hyperref}
% Note: As of 2008, this is NOT RECOMMENDED.  If you use it, check its effects carefully.

\title{Cohomology and Support Varieties\\
  for Classical Lie Superalgebras}

\author{Andrew B. Maurer}

\previousdegrees{B.S., University of Massachusetts, 2014}

\thisdegree{Doctor of Philosophy}  % or Doctor of Philosophy, etc.

\thesistype{Dissertation}     % or Dissertation

\thesismonth{May}       % put the graduation month here, not today's month...
\thesisyear{2019}

\professor{Daniel Nakano}
%\secondprofessor{Alonzo Church}   % Only if you have 2 major professors!

\fulltitle{Support Varieties for Classical Lie Superalgebras}

\indexwords{Cohomology, Lie~Superalgebra, Geometric~Invariant~Theory, Representation~Theory, Support~Varieties, Algebraic~Geometry, Homological~Algebra}

\dean{Suzanne Barbour}

\memberi{Brian Boe}
\memberii{William Graham}
\memberiii{Laura Rider}
% Use \memberiii, \memberiv, \memberv for up to 3 more members if needed.

\begin{document}
\pagenumbering{roman}

\begin{abstract}
  Classical Lie superalgebras arise in the physical theory of supersymmetry and behave analogously to Lie algebras in the theory of algebraic groups. In the theory of Lie algebras, the Koszul resolution is finite, meaning relative cohomology rings are finite as a vector over the field. For Lie superalgebras, the Koszul resolution is infinite. A theorem of Boe-Kujawa-Nakano states that for classical Lie superalgebras, the cohomology ring relative to the even component has finite Krull dimension, and the structure of this cohomology ring is determined by the invariant theory of a reductive group's action on a vector space. This realization opens the door to the study of the support variety theory for classical Lie superalgebras.
  
  The main result of this thesis is a generalization of the result of Boe-Kujawa-Nakano. The main theorem asserts that the cohomology ring of a classical Lie superalgebra relative to any even subsuperalgebra has finite Krull dimension, and is indeed a finite extension of a subquotient of the Boe-Kujawa-Nakano cohomology ring via restriction. The proof of the main theorem relies on a spectral sequence inspired by that of Hochschild-Serre.

  The spectral sequence used to prove finite generation proves to be an invaluable tool in analyzing the behavior of cohomology rings. An example is presented in which the Krull dimension of a relative cohomology is positive but not equal to the Krull dimension of Boe-Kujawa-Nakano cohomology. Conditions are given for when the cohomology ring will be Cohen-Macaulay.

  With finite-generation established, the final chapter of this dissertation is devoted to studying the relative support variety theory for modules. A realization morphism induced by restriction of functions plays a role similar to that of Friedlander-Parshall's realization morphism for restricted Lie algebras. The main goal of this chapter is to work towards a conjectural tensor product theorem for Lie superalgebras.

%   This is the abstract, a brief summary of the contents of the thesis.  It
% is limited to 150 words in length for a master's thesis or 350 words for
% a doctoral dissertation.

% The abstract page(s) are not numbered and are not necessarily included
% in the bound copies.  Likewise, the signature page is not counted in
% page numbering because not all copies contain it.

% Throughout this sample thesis, {\bf please note
% that the layout obtained with \LaTeX\ is not meant to be a
% perfect duplicate of the Microsoft Word examples in the \emph{Graduate
% School Style Manual.}}  \LaTeX\ has additional typographic tools at its
% disposal, such as {\sc small capitals} and various subtle adjustments
% of spacing, which are used by the \LaTeX\ UGa style sheet in accordance
% with the standard practices of the book-printing industry.

% The index words at the bottom of the abstract should be chosen carefully,
% preferably with the help of one or two of your colleagues.
% They are the words by which people will find your thesis when searching
% the scientific literature.
% If you want to get credit for your ideas, be sure to choose a good set of
% index words so that people doing related work will know about yours.
\end{abstract}

\maketitle    % Creates title page, copyright page if any, and approval page.


% \chapter*{Dedication}

% The Dedication is optional, but if it is included, it should have
% a roman numeral page number but not be included in the table of
% contents.  To achieve that, we declare it as a \verb"\chapter*" in \LaTeX.


\pseudochapter{Acknowledgments}

The author would like to thank his advisor, Dr Daniel Nakano for much motivation and support, and Dr William Graham for helpful conversations involving spectral sequences and Lie algebra cohomology. This work was partially supported by the Research and Training Group in Algebraic Geometry, Algebra and Number Theory grant DMS-1344994 funded by the National Science Foundation.

% In a real thesis, this section would contain acknowledgments such
% as, ``This work was funded by National Science Foundation Grant
% Number AAA-00-00000 (Benjamin Franklin, Principal Investigator),''
% and ``I would like to thank John Doe for helping me proofread my
% thesis and Mary Roe for drawing my graphs.''

% The acknowledgments are included in the table of contents but do
% not have a chapter number.  To achieve that, we
% declare them to be a \verb"\pseudochapter" (which is defined only in \verb"uga.sty").

% I would like to thank John Carroll for first pointing out to me that
% Georgia needed --- but did not have --- a \LaTeX\ style sheet.
% Immediately I looked at the widely distributed Stanford, MIT, and
% University of Illinois style sheets and decided they would not do at
% all. And that's how I got into this.

% I would also like to thank Mark Maloof for helping me find some bugs in
% the original version.

% Zachary Hunter deserves special thanks for revising the UGA style
% file to work with the new version of \LaTeX\ (2e) as well as the
% previous version (2.09).

% The revisions in
% Version 3.0 of \verb"uga.sty" are
% largely the work of Isidor Ruderfer.


% \pseudochapter{Preface}

% If your thesis has a preface, this is where it goes.
% A preface is not an introduction, and most theses do not need them.


\tableofcontents

%\listoffigures   % Optional - Omit this line if you don't want a list of figures.
%\listoftables    % Optional - Omit this line if you don't want a list of tables.

\newpage
\pagenumbering{arabic}  % Ordinary pages have Arabic numerals.

\chapter{Introduction}
\label{chap:intro}

\section{Motivation}

Establishing finite generation of cohomology rings is a powerful result in representation theory which links cohomology theory with commutative algebra and algebraic geometry. For example, Evens \cite{Evens-cohomology-ring} and Venkov \cite{MR0108788} each independently proved that the cohomology ring of a finite group is finitely generated. This result was used by Quillen \cite{MR0298694}, Carlson \cite{MR723070}, Chouinard \cite{MR0401943}, and Alperin-Evens \cite{MR621284} to study the cohomology variety of the finite group. This allowed those listed, among others, to use techniques from classical algebraic geometry in the study of representation theory of finite groups. Similar work has been carried out in other contexts; by Friedlander-Parshall \cite{FP-unipotent,MR860682} for restricted Lie algebras, and by Friedlander-Suslin \cite{MR1427618} for finite-dimensional cocommutative Hopf algebras.

Relative cohomology, as defined by Hochschild \cite{hochschild} is less understood than ordinary cohomology. For instance, the cohomology ring of a finite group relative to a subgroup need not be finitely generated. Indeed, Brown \cite{MR1310744} provided an example of a finite group whose relative cohomology is infinitely generated. Surprisingly, in the case $\g = \even{\g} \oplus \odd{\g}$ is a (finite-dimensional) classical Lie superalgebra, the cohomology ring of $\g$ relative to $\even{\g}$ is always finitely generated. Specifically, Boe-Kujawa-Nakano \cite{BKN-1} realized this relative cohomology ring as the invariants of a polynomial ring under the action of a reductive group. In fact, in the case of Lie superalgebras, ordinary cohomology is often times finite-dimensional as a vector space, as proved by Fuks-Leites \cite{fuks1984cohomology}
% \cite{MR1450424}
. This implies relative cohomology rings carry more representation theoretic information than their ordinary counterparts. Furthermore, Boe-Kujawa-Nakano \cite{BKN-1} demonstrated the atypicality of a supermodule -- a combinatorial invariant defined by Kac-Wakimoto \cite{MR1327543} -- is realized as the dimension of the support variety of that module. The geometrization of combinatorial ideas makes support variety theory useful and powerful.

One of the main results of this paper asserts that for a classical Lie superalgebra, cohomology rings relative to even subalgebras are finitely-generated over $\CC$, and the relative cohomology of a finite-dimensional module is a Noetherian module for this ring. In proving the main theorem, a spectral sequence is constructed  which relates relative Lie algebra cohomology to odd degree elements of the Lie superalgebra in an interesting way. The main theorem paves the way to define and investigate support varieties for supermodules relative to a broader class of subalgebras. The importance of this result is apparent in that cohomology relative to an even subalgebra provides a middle ground between the case of absolute cohomology of Fuks-Leites and cohomology relative to $\even{\g}$ of \cite{BKN-1}.

\section{Overview of Dissertation}
\label{sec:overview}

\TODO{Revamp}{This was written for my paper. Once dissertation is completed I will rework this for the contents of the dissertation.}

Let $\g = \even{\g} \oplus \odd{\g}$ be a Lie superalgebra, $\a \leq \even{\g}$ a subalgebra, and $M$ a finite-dimensional $\g$-module. The theory of relative cohomology \cite{hochschild} can be used to define relative cohomology groups $\H^n(\g,\a;M)$, which may be viewed as relative derived functors of $\Hom_{(\g,\a)}(\CC,-)$. In \cite[Theorem 2.5.2]{BKN-1} it was shown that when $\g$ is a classical Lie superalgebra and $\a = \even{\g}$, the cohomology ring $\H^\bullet(\g,\even{\g};\CC)$ is the subring $S(\odd{\g}^*)^{\even{G}}$ of invariants under a reductive group action, and is thus finitely generated over $\CC$. This paper's main result extends this work to arbitrary subalgebras $\a \leq \even{\g}$.

\begin{maintheorem}
  Let $\g = \even{\g} \oplus \odd{\g}$ be a classical Lie superalgebra, and $\a \leq \even{\g}$ an (even) subalgebra, and $M$ a $\g$-module.
  \begin{enumerate}[\indent\rm (a)]
    \item There is a spectral sequence $\{E_r^{p,q}\}$ which computes cohomology and satisfies
  \[
    E_2^{p,q}(M) \cong \H^p(\g,\even{\g};M) \otimes \H^q(\even{\g},\a;\CC) \Rightarrow \H^{p+q}(\g,\a;M)
  \]
  For $1 \leq r \leq \infty$, $E_r^{\bullet,\bullet}(M)$ is a module for $E_2^{\bullet,\bullet}(\CC)$. When $M$ is finite-dimensional, $E_2^{\bullet,\bullet}(M)$ is a Noetherian $E_2^{\bullet,\bullet}(\CC)$-module.
    \item Moreover, the cohomology ring $\H^\bullet(\g,\a;\CC)$ is a finitely-generated $\CC$-algebra.
  \end{enumerate}
\end{maintheorem}

The paper is outlined as follows. In Section \ref{sec:prelims}, Lie superalgebras, modules for Lie superalgebras, and cohomology of Lie superalgebras are defined. The pace is brisk and the interested reader will find a more thorough overview in \cite{BKN-1,Kac}. In Section \ref{sec:spectral-sequence} the author establishes finite generation of the relative cohomology ring. To do so, a first-quadrant spectral sequence as described above is constructed, similar to that of Hochschild and Serre \cite{HS-53}, pages are identified, and a standard argument is used. Additionally, the edge homomorphism of the $E_2^{\bullet,0} \to E_\infty^{\bullet,0}$ is identified as restriction, making $\H^\bullet(\g,\a;\CC)$ an integral extension of a homomorphic image of $\H^\bullet(\g,\even{\g};\CC)$. Equipped with the spectral sequence of the previous section, we devote Section \ref{sec:structure} to investigating the structure of these relative cohomology rings. For the relative cohomology ring to be Cohen-Macaulay, it is shown to be sufficient that the spectral sequence of Section \ref{sec:spectral-sequence} collapse at the $E_2$ page. This is used to compute a broad class of examples. Finally, we are in a position to systematically study support varieties for Lie superalgebras, which we do in Section \ref{sec:support-varieties}. In this section, support varieties are defined and several basic properties are stated before addressing the more difficult questions of realizability and connectedness. Our realizability theorem demonstrates a naturality between support varieties for $(\g,\even{\g})$ and those for $(\g,\a)$.

The author would like to thank his advisor, Dr Daniel Nakano for much motivation and support, and Dr William Graham for helpful conversations involving spectral sequences and Lie algebra cohomology. This work was completed as a part of the author's dissertation at University of Georgia, and was partially supported by the Research and Training Group in Algebraic Geometry, Algebra and Number Theory grant DMS-1344994 funded by the National Science Foundation.

%%% Local Variables:
%%% TeX-master: "../dissertation.tex"
%%% End:
\chapter{Algebraic Geometry}
\label{chap:ag-git}

\section{Motivation}
\label{sec:motivation-ag}

Algebraic geometry is a remarkably powerful discipline which allows one to view the hidden geometry arising in commutative ring theory. It turns out, many of these techniques that arose from commutative ring theory hold equally well for graded-commutative rings, i.e., graded rings satisfying $a \cdot b = (-1)^{\bar a \cdot \bar b} b \cdot a$. This is handy for representation theorists because cohomology rings are very often graded-commutative.

Algebraic geometry behaves nicely when the rings in question are finitely generated over a ground field, in which case the associated geometric object is a variety.

\section{The Spectrum of a Ring}
\label{sec:spec}

\begin{definition}
  Let $\H^\bullet$ be a graded-commutative ring, which is finitely generated as a $\CC$-algebra. The \emph{spectrum} of $\H^\bullet$ is the set:
  \[
\Spec(\H^\bullet) = \left\{ [\p] \mid \p \leq \H^\bullet \text{ is a prime ideal} \right\}
  \]
  The set $\Spec(\H^\bullet)$ is equipped with a topology whose closed sets are the vanishing sets of ideals
  \[
    \Z(I) = \left\{ [\p] \in \Spec(\H^\bullet) \mid I \subseteq \p \right\}
  \]
\end{definition}

\begin{note}
  Despite the fact that $\H^\bullet$ is a graded ring, we choose to use $\Spec$ and not $\Proj$. As a result, the varieties considered are conical affine varieties and not projective varieties. This is especially useful because
  \[
    \Hom_\text{ring}(\H^\bullet_1,\H^\bullet_2) \longleftrightarrow \Mor_\text{var}(\Spec(\H^\bullet_2),\Spec(\H^\bullet_1))
  \]
  
\end{note}

In what follows, we will primarily be concerned with the topological space $\Spec(\H^\bullet)$, not with the sheaf of rings associated to it. The interested reader may consult Chapter II of Hartshorne\cite{hartshorne}.

\begin{proposition}
  Let $\H^\bullet_1$ and $\H^\bullet_2$ be commutative or graded-commutative rings. Let $I_1 \leq \H^\bullet_1$ be an ideal.
  \begin{enumerate}[(a)]
  \item A homomorphism of rings $\varphi: \H^\bullet_1 \to \H^\bullet_2$ corresponds to a morphism of varieties $\varphi^*: \Spec(\H^\bullet_2) \to \Spec(\H^\bullet_1)$. This map is determined on points by $\varphi^*([\p_2]) = \varphi^{-1}(\p_2) \leq \H^\bullet_1$.
  \item The spectrum of a quotient $\Spec(\H^\bullet_1 / I_1)$ is isomorphic to $\Z(I_1)$. Furthermore, the quotient morphism $\pi : \H^\bullet_1 \onto \H^\bullet_1 / I_1$ corresponds to the inclusion morphism $\pi^*: \Z(I) \hookrightarrow \Spec \H^\bullet_1$.
  \end{enumerate}
\end{proposition}


\section{Modules and Subvarieties}
\label{sec:mods}

As above, consider a graded-commutative ring $\H^\bullet$. Let $\M$ be a finitely-generated $\H^\bullet$-module. This module has an \emph{annihilator}
\[
  \Ann_{\H^\bullet}(\M) = \{x \in \H^\bullet \mid x . M = 0 \}
\]

\begin{definition}
  The \emph{support} of $\M$ is the vanishing set of the annihilator of $\M$:
  \[
\Supp_{\H^\bullet}(\M) = \Z\left(\Ann_{\H^\bullet}(\M)\right) \subseteq \Spec(\H^\bullet)
  \]
  
\end{definition}

\begin{proposition}
  Let $\H^\bullet$ be a commutative or graded-commutative ring, and let $\M_i$ be $\H^\bullet$-modules.
  \begin{enumerate}[(a)]
  \item If $\H^\bullet$ is graded-commutative, then $\Supp_{\H^\bullet}(\M)$ is closed and conical.
  \item $\Supp_{\H^\bullet}(\M_1 \oplus \M_2) = \Supp_{\H^\bullet}(\M_1) \cup \Supp_{\H^\bullet}(\M_2)$
  \item If $0 \to \M_1 \to \M_2 \to \M_3 \to 0$ is a short exact sequence, and $\sigma \in S_3$ is a permutation of $\{1,2,3\}$, then
    \[
      \Supp_{\H^\bullet}(\M_{\sigma(1)}) \subseteq \Supp_{\H^\bullet}(\M_{\sigma(2)}) \cup \Supp_{\H^\bullet}(\M_{\sigma(3)})
    \]
    
  \end{enumerate}
\end{proposition}

\begin{proof}
  \begin{enumerate}[(a)]
  \item This is immediate. Vanishing sets are closed by definition. Because $\Supp_{\H^\bullet}(\M)$ is the vanishing of a homogeneous ideal, it must be conical.
  \item 
  \item 
  \end{enumerate}
\end{proof}


%%% Local Variables:
%%% TeX-master: "../dissertation.tex"
%%% End:
\chapter{Algebraic Groups}

\section{Overview}
\label{sec:grp-motivation}

This section's aim is to define reductive algebraic groups and introduce results important for what follows. The most relevant fact about reductive groups is Proposition \ref{prop:fg-coordinate}, which states the ring of invariants of a reductive algebraic group's action on the coordinate ring of an affine algebraic variety is finitely generated. Lie algebras are introduced as the tangent space at the identity to an affine algebraic group. Morphisms of affine algebraic groups induce morphisms of Lie algebras. This fact leads to the conclusion that group actions differentiate to Lie algebra actions.

The main reference for this chapter is Humphreys \cite{MR0396773} or Jantzen \cite{MR2015057}.

\section{Algebraic Groups ABCs}
\label{sec:basics-of-groups}

\begin{definition}
  An \emph{algebraic group} is a complex affine algebraic variety $G$ equipped with an identity element $e: \Spec(\CC) \to G$, multiplication morphism $m : G \times G \to G$ and an inverse morphism $i: G \to G$. These morphisms satisfy the typical group-theoretic axioms, expressed as commutative diagrams.

  An algebraic group is \emph{linear} if it is a subgroup of $\GL(n)$, the set of ivertible linear matrices.
\end{definition}

\begin{example} \label{ex:lag}
  The following are examples of linear algebraic groups:
  \begin{enumerate}
  \item $\GG_a$, the additive group $\CC^+$. This can be embedded into $2 \times 2$ matrices via the embedding
    \[
      a \mapsto
      \begin{pmatrix}
        1 & a \\
        0 & 1
      \end{pmatrix}
    \]
  \item The \emph{general linear group} $\GL(n)$, the set of invertible $n \times n$ matrices. When $n = 1$, we usually denote this $\GG_m = \GL(1)$, the multiplicative group $\CC^\times$.
  \item The \emph{special linear group}, $\SL(n)$, consisting of invertible matrices with determinant equal to 1.
  \item The \emph{orthogonal matrix group} $O(n)$, consisting of invertible matrices such that $M \cdot M^T = 1$. This is not connected in the Zariski topology, as the morphism $\det: O(n) \to \CC^\times$ shows. The \emph{special orthogonal group} $SO(n) = O(n) \cap \SL(n)$ is connected.
  \end{enumerate}
\end{example}

\section{Modules for Algebraic Groups}
\label{sec:group-modules}

As in the theory of finite groups, algebraic groups are best understood through their actions on vector spaces. This theory is made complete by studying the category of representations of $G$, i.e., the set of modules for $G$ with morphisms which commute with the action of $G$.

\begin{definition}
  Let $\even{G}$ be a linear algebraic group. A \emph{representation} of $\even{G}$ is a complex vector space $V$ with a morphism of algebraic varieties $\rho: \even{G} \to \GL(V)$, and elements of $G$ act on elements of $V$ via the map $g.v = \rho(g)(v)$. Alternatively, a representation may be called a \emph{$\even{G}$-module}.
\end{definition}

The following are the most important examples of representations for algebraic groups.

\begin{example}
  \begin{enumerate}
  \item The trivial representation $\even{G} \to \GG_m$ defined by $g \mapsto 1$.
  \item The standard representation of $\even{G} \subseteq \GL(n)$ acts on $\CC^n$ via $g.v = g(v)$.
  \item The determinant of a representation $\even{G} \to \GL(n) \to \GG_m$ sending $g \mapsto \rho(g) \mapsto \det(\rho(g))$.
  \item If $V$ is a representation of $\even{G}$, then so is $V^* = \Hom_\CC(V,\CC)$. This is via the action $g.f = f_g \in V^*$, where $f_g(v) = f(g^{-1}.v)$
  \item If $V$ is a representation, then so is $V^{\otimes m}$ via the diagonal action $g.(v_1 \otimes \ldots \otimes v_m) = g.v_1 \otimes g.v_2 \ldots \otimes g.v_m$. This action extends to $\even{G} \actson \T^\bullet(V)$.
  \item The symmetric and exterior products of a representation $V$ are again representations, with action inherited from the action on the tensor power.
  \end{enumerate}
\end{example}

\section{Reductive Algebraic Groups}
\label{sec:reductive}

This section introduces the notion of reductivity. Reductive algebraic groups are important because they behave particularly well with respect to invariant theory, which proves to be an invaluable tool in the computation of cohomology rings for Lie superalgebras (see Chapter~\ref{chap:rel-cohom}).

\begin{definition}
  \begin{enumerate}
  \item An element $N$ in a linear algebraic group is \emph{unipotent} if $N- I_n$ is a nilpotent matrix.
  \item The \emph{radical} of a linear algebraic group is the maximal connected, normal, solvable subgroup.
  \item The \emph{unipotent radical} of a linear algebraic group $\even{G}$ is the set of unipotent elements in the radical of $\even{G}$.
  \end{enumerate}
\end{definition}
The previous definition is a bit obscure. To see how this acts in context, refer to the following examples in which unipotent radicals are computed for common algebraic groups.
\begin{example}
  \begin{enumerate}
  \item Let $\even{G} = \GL(n)$. Up to conjugation, the radical of $G$ is the set of diagonal matrices, isomorphic to $\GG_m^n$. The only unipotent element in the radical is the identity. Thus, the unipotent radical of $\even{G}$ is $\{I_n\}$.
  \item Let $\even{G} = \SL(n)$. Up to conjugation, the radical is the set of diagonal matrices of determinant 1, isomorphic to $\GG_m^{n-1}$. The only unipotent element of this radical is the identity. The unipotent radical is again trivial.
  \item Consider the group $\GG_a$. Use the embedding of Example \ref{ex:lag} and observe that every element of $\GG_a$ is unipotent. Now notice $\GG_a$ itself is connected, normal, and solvable. Thus $\GG_a$ is its own unipotent radical.
  \end{enumerate}
\end{example}

It is a fact that in many well-behaved cases, the unipotent radical is trivial. The word for this behavior is \emph{reductive}, and proves to be of paramount importnace in representation theory and algebraic geometry.

\begin{definition}
  A linear algebraic group is \emph{reductive} if its unipotent radical is trivial. 
\end{definition}

Reductive groups have the following vitally important property, attributed to Hilbert. The main application of this theorem is to the cohomology ring of a classical Lie superalgebra relative to its even subsuperalgebra, and it turns out this ring is the invariants of a polynomial ring under a reductive group action.

\begin{proposition} \label{prop:fg-coordinate}
  Let $\even{G}$ be a reductive group acting on an affine algebraic variety $X$. The algebra of invariants $\CC[X]^{\even{G}}$ is finitely-generated over $\CC$.
\end{proposition}
\begin{corollary}
  Let $\even{G}$ be a reductive group acting on a vector space $\odd{\g}$. The ring of polynomial invariants $S^\bullet(\odd{\g}^*)^{\even{G}}$ is a finitely-generated $\CC$-algebra.
\end{corollary}

\section{The Lie Algebra of an Algebraic Group}
\label{sec:lie-algebra}

This section covers a functor $\Lie$ which maps algebraic groups to their Lie algebras. Mainly, we are interested in the way an action $\even{G} \actson V$ induces an action of $\Lie(\even{G}) = \even{\g} \actson V$.

\begin{definition}
  Let $X \subseteq \AA^n$ be an affine algebraic variety containing a point $P = \Z(\p)$, and let $\O_P = \CC[X]_\p$. The \emph{Zariski tangent space} $\T_P(X)$ is defined to be
  \begin{equation}
    \label{eq:tangent-space}
    \T_P(X) = \left(\p \O_P / \p^2 \O_P\right)^*.
  \end{equation}
  When $X = \even{G}$ is an affine algebraic group, the \emph{Lie algebra of $\even{G}$} is the tangent space at the identity,
  \begin{equation}
    \label{eq:lie-alg}
    \even{\g} = \Lie(\even{G}) = \T_1(\even{G}).
  \end{equation}
\end{definition}

\begin{proposition} \label{prop:lie-alg}
  The Lie algebra $\even{\g}$ of an affine algebraic group $\even{G}$ inherits a bilinear bracket operation $[\cdot,\cdot]: \even{\g} \otimes \even{\g} \to \even{\g}$ which satisfies the following two axioms:
  \begin{enumerate}[(L1)]
  \item $[x,x] = 0$
  \item $[x,[y,z]] = [[x,y],z] + [y,[x,z]]$
  \end{enumerate}
  Additionally, the construction of a Lie algebra is functorial, meaning a morphism $\varphi:\even{G} \to \even{G}'$ to a morphism $d\varphi:\even{\g} \to \even{\g}'$ which respects the bracket operation.
\end{proposition}

Let us introduce several important examples of Lie algebras. 

\begin{example}
  \begin{enumerate}
  \item If $\even{G}$ is an Abelian group, then $\even{\g}$ is an \emph{Abelian Lie algebra}, meaning $[x,y] = 0$ for every $x,y \in \g$.
  \item If $\even{G} = \GL(n)$, then $\even{\g} = \gl(n)$, called the \emph{general linear Lie algebra}. As a set, $\gl(n)$ is all $n \times n$ matrices with bracket given by $[A,B] = AB - BA$.
  \item If $\even{G} = \SL(n)$, then $\even{\g} = \sl(n)$, called the \emph{special linear Lie algebra}. This is the Lie subalgebra of $\gl(n)$ consisting of trace zero matrices.
  \end{enumerate}
\end{example}

When $\even{G}$ acts on a vector space $V$, this amounts to a morphism $\even{G} \to \GL(V)$. As we have seen in Proposition \ref{prop:lie-alg}, this yields a morphism of Lie algebras $\even{\g} \to \gl(V)$. This fact is noted in the following proposition.

\begin{proposition}
  The $\even{G}$-module structure on a vector space $\odd{\g}$ induces the structure of a $\Lie(\even{G}) = \even{\g}$-module structure on $\odd{\g}$.
\end{proposition}

%%% Local Variables:
%%% TeX-master: "../dissertation.tex"
%%% End:
\chapter{Lie Superalgebras}
\label{chap:lsa}

\section{Motivation}
\label{sec:motivate-lsa}

A \emph{Lie superalgebra} is a $\ZZ_2$-graded analogue of a Lie algebra. Lie superalgebras originated in the physical theory of \emph{supersymmetry} and play a similar role as Lie algebras, in that they arise as tangent spaces to \emph{Lie supergroups} at the identity element.

A thorough overview of Lie superalgebra theory is provided by Victor Kac \cite{Kac}. A main result of this paper is the classification of simple classical Lie superalgebras.

\begin{theorem}[Kac, \cite{Kac}]
  A simple classical Lie superalgebra is isomorphic to either to one of the simple Lie algebras $A_n$, $B_n$, \ldots, $E_8$ or to one of $A(m,n)$, $B(m,n)$, $C(n)$, $D(m,n)$, $D(2,1;\alpha)$, $F(4)$, $G(3)$, $P(n)$, or $Q(n)$.
\end{theorem}

\section{Definition and Examples}

We start this chapter with a definition.

\begin{definition}
  A \emph{superspace} is a $\ZZ_2$-graded complex vector space $V = \even{V} \oplus \odd{V}$. An element of $V_i$ is called \emph{homogenous of degree $i$}. The \emph{superdimension} of $V$ is the ordered pair $\sdim V = (\dim \even{V}, \dim \odd{V})$.
\end{definition}

It turns out that the vector space of homomorphisms also has the natural structure of a superspace.

\begin{example} \label{ex:hom}
  Let $V$ and $W$ be superspaces. The vector space of linear homomorphisms $\Hom_\CC(V,W)$ is naturally a superspace:
  \begin{align*}
    \even{\Hom_\CC(V,W)} &= \{ \varphi \mid \varphi(V_i) \subseteq \varphi(W_i)\} \\
    \odd{\Hom_\CC(V,W)} &= \{ \varphi \mid \varphi(V_i) \subseteq \varphi(W_{i + \bar 1})\}
  \end{align*}
  In this way,
  \[
    \Hom_\CC(V,W) = \even{\Hom_\CC(V,W)} \oplus \odd{\Hom_\CC(V,W)}
  \]
  
\end{example}

With the notion of superspace defined, we may now define the concept of a Lie superalgebra. This is an algebraic object which, in the theory of supersymmetry \cite{varadarajan}, plays the role analogous to that of a Lie algebra in representation theory of algebraic groups. In supersymmetry, formulas are typically defined on homogeneous elements and extended by linearity. Additionally, commutation of two homogeneous quantitites results in an additional factor of $(-1)$ to the product of their degrees.

\begin{definition}
  A \emph{Lie superalgebra} is a superspace $\g = \even{\g} \oplus \odd{\g}$, equipped with a bilinear bracket $[\cdot,\cdot]: \g \otimes \g \to \g$ satisfying the following two properties:
  \begin{enumerate}
  \item [(S1)] For $x,y,z$ homogeneous elements of $\g$,
    \[
      [x,[y,z]] = [[x,y],z] + (-1)^{\bar x \cdot \bar y} [y, [x,z]]
    \]
    
  \item [(S2)] For $x,y$ homogeneous elements of $\g$,
    \[
      [x,y] + (-1)^{\bar x \cdot \bar y} [y,x] = 0
    \]
    
  \end{enumerate}
\end{definition}

It is worth noting that the even subsuperalgebra $\even{\g}$ is, in fact, a Lie algebra. Furthermore, the subset of odd elements $\odd{\g}$ is a module for the Lie algebra $\even{\g}$.

\begin{example}
  \begin{enumerate}
    \item Let $V = \even{V} \oplus \odd{V}$ be a superspace of superdimension $(m|n)$. The \emph{general linear Lie superalgebra} $\gl(V)$ or $\gl(m|n)$ is the superspace $\Hom_\CC(V,V)$, with grading of Example \ref{ex:hom}, visualized as
  \[
    \even{\gl(m|n)} =
    \left(
      \begin{array}{c|c}
        A_{m \times m} & 0 \\
        \hline
        0 & A_{n \times n}
      \end{array}
    \right)
    \text{ and }
    \odd{\gl(m|n)} =
    \left(
      \begin{array}{c|c}
        0 & A_{m \times n} \\
        \hline
        A_{n \times m} & 0
      \end{array}
    \right)
  \]
  The bracket operation on $\gl(m|n)$ is defined for homogeneous elements via
  \[
    [A,B] = A \cdot B - (-1)^{\bar A \cdot \bar B} B \cdot A
  \]
  \item Consider the matrix $A \in \gl(m|n)$, decomposed as
  \[
    A = 
    \left(
      \begin{array}{c|c}
        A_{m \times m} & A_{m \times n} \\
        \hline
        A_{n \times m} & A_{n \times n}
      \end{array}
    \right)
  \]
  The \emph{supertrace of $A$} is $\sTr(A) = \Tr(A_{m \times m}) - \Tr(A_{n \times n})$. The \emph{special linear Lie superalgebra} is $\sl(V)$ or $\sl(m|n)$ and consists of all matrices in $\gl(V)$ with supertrace 0, i.e.,
  \[
    \sl(m|n) = \left\{ A \in \gl(m|n) \mid \sTr(A) = 0 \right\}
  \]
\end{enumerate}
\end{example}

\begin{example}
  \label{ex:q1}
  Let $\g$ be a Lie superalgebra. We will classify all subsuperalgebras generated by a single homogeneous element $x \in \g$.
  \begin{enumerate}
  \item If $x \in \even{\g}$, then $[x,x] = 0$. As such $\left<x\right>$ is a one-dimensional simple Lie algebra $\g \cong \CC \oplus \{0\}$
  \item If $x \in \odd{\g}$, and $[x,x] = 0$ then there are no even elements and thus $\g \cong 0 \oplus \CC$.
  \item \label{ex:q13} If $x \in \odd{\g}$ and $[x,x] = y \neq 0$, then the super Jacobi axiom says $[x,y] = [x,[x,x]] = [[x,x],x] - [x,[x,x]]$. Applying super anticommutativity yields $[x,y] = 0$. The multiplication table for this Lie superalgebra is presented in Figure \ref{fig:q1-mult}. Lie superalgebras isomorphic to this one are referred to as \emph{of type $\mathbf{q}(1)$}.
    \begin{figure}[h]
      \label{fig:q1-mult}
      \centering
      \begin{tabular}{r|| c | c |}
$[\cdot,\cdot]$ & y & x \\
\hline\hline
        y & x & 0 \\
        x & 0 & 0 \\
\hline
\end{tabular}
      \caption{Multiplication table for $\mathbf{q}(1)$}
    \end{figure}
  \end{enumerate}
\end{example}

\section{Classical Lie Superalgebras}
\label{sec:classical-superalgebras}

This section introduces a broad class of Lie superalgebras whose structure is governed by the theory of reductive algebraic groups. Later, we will see that the cohomology theory of these Lie superalgebras is also determined by the invariant theory, which behaves particularly nicely. 

\begin{definition}
  A \emph{classical Lie superalgebra} is a Lie superalgebra $\g = \even{\g} \oplus \odd{\g}$ such that there exists a reductive algebraic group $\even{G}$ which acts on $\odd{\g}$ which satisfies
  \begin{enumerate}
  \item $\even{\g} = \Lie(\even{G})$
  \item The action of $\even{G} \actson \odd{\g}$ differentiates to yield the adjoint action $\even{\g} \actson \odd{\g}$.
  \end{enumerate}
\end{definition}

\begin{example}[Lie superalgebra of type $\q(n)$, as in {\cite[\S 8.3]{BKN-1}} ] \label{ex:q(n)}
  We define a Lie superalgebra called $\q(n)$ as a Lie subsuperalgebra of the special linear Lie superalgebra $\q (n) \leq \sl(n+1 \mid n+1)$.
  \[
    \q(n) =
    \left\{
    \left(
    \left.    
      \begin{array}{c|c}
        A & B \\
        \hline
        B & A
      \end{array}
    \right)
    \right| A,B \in \M_{n \times n}(\CC)
    \right\}
  \]
  A quick computation shows that $\dim_s \q(n) = (n^2 \mid n^2)$ (and therefore $\dim_\CC \q(n) = 2n^2$), $\even{\q(n)} \cong \gl(n)$, $\odd{\q(n)} \cong \gl(n)$, and $\odd{\g}$ is the adjoint representation of $\even{\g}$. In this way, $\q(n)$ is a classical Lie superalgebra with $\even{G} = \GL(n)$, and $\even{G} \actson \odd{\g}$ via conjugation, yielding the adjoint action of $\even{\g} \actson \odd{\g}$.

  Additionally, we may verify the Lie superalgebra of Example \ref{ex:q1} Part \ref{ex:q13} is indeed the classical Lie superalgebra described above. This follows by taking a basis of the form
  \[
    x = 
        \left(
      \begin{array}{c|c}
        0 & 1 \\
        \hline
        1 & 0
      \end{array}
    \right)
    \text{ and }
    y = 
        \left(
      \begin{array}{c|c}
        1 & 0 \\
        \hline
        0 & 1
      \end{array}
    \right)
  \]
  and verifying that the multiplication table of Figure \ref{fig:q1-mult} is valid.
\end{example}


\section{Modules for Lie Superalgebras}
\label{sec:modules}

As with any object in abstract algebra, we care not simply about Lie superalgebras on their own, but also about their actions on vector spaces. Because of the grading on $\U(\g)$ (introduced in Section \ref{sec:universal-enveloping-superalgebras}), we require $\g$-modules to be graded $M = \even{M} \oplus \odd{M}$. With this requirement, the category of $\g$-modules is no longer Abelian. In order to make use of the tools of homological algebra, we make use of the subcategory whose objects are $\g$-modules and whose morphisms are \emph{even} homomorphisms of $\g$-modules. This subcategory is useful when the \emph{parity change functor} $\Pi$ is used, in which case all data contained in the catogory of $\g$-modules may be recovered.

\begin{definition}
  A module may be defined in the following three equivalent ways, each of which is useful in certain cases.
  \begin{enumerate}
  \item $V = \even{V} \oplus \odd{V}$ is a graded module for the universal enveloping superalgebra $\U(\g)$ (to be defined in \ref{sec:universal-enveloping-superalgebras}).
  \item $V = \even{V} \oplus \odd{V}$ is a graded complex vector space and $\rho : \g \to \gl(V)$ is an even homomorphism of vector spaces. The action is $g.v = \rho(g)(v)$.
  \item $V = \even{V} \oplus \odd{V}$ is a graded complex vector space and $\g$ acts on $V$ in a linear fashion such that the following condition holds:
\begin{equation}
  g_1.(g_2.x) - (-1)^{\bar g_1 \bar g_2} g_2. (g_1.x) = [g_1,g_2].x
\end{equation}
for all homogeneous $g_1, g_2 \in \g$.
  \end{enumerate}
\end{definition}

\begin{definition}
  A \emph{homomorphism} of $\g$-modules $f: M \to N$ is a homogeneous linear map (i.e., $f \in \even{\Hom(M,N)} \cup \odd{\Hom(M,N)}$) satisfying the following property:
  \[
    f(g.m) = (-1)^{\bar f \cdot \bar g} g.f(m)
  \]
  for $g \in \g$, $m \in M$.
\end{definition}

Unfortunately, the category of $\g$-modules is not an Abelian category. We remedy this situation by considering the even subcategory, whose objects are $\g$-modules and whose morphisms are even homomorphisms $\even{\Hom_\g(M,N)}$.

\begin{proposition}
  The category $\even{\Mod(\g)}$ is an Abelian category.
\end{proposition}

\begin{definition}
  The \emph{parity change functor} is a functor $\Pi: \Mod(\g) \to \Mod(\g)$ which switches the grading of modules. Symbolically $\even{\Pi(M)} = \odd{M}$, and $\odd{\Pi(M)} = \even{M}$.
\end{definition}

\begin{proposition}
  Let $V = \even{V} \oplus \odd{V}$ be a superspace. Then $\gl(V)$ is naturally isomorphic to $\gl(\Pi(V))$. This isomorphism may be visualized as follows:
\begin{equation}
    \left(
      \begin{array}{c|c}
        A_{n \times n} & B_{n \times m} \\
        \hline
        C_{m \times n} & D_{m \times m}
      \end{array}
    \right)
    \mapsto
    \left(
      \begin{array}{c|c}
        D_{m \times m} & C_{m \times n} \\
        \hline
        B_{n \times m} & A_{n \times n}
      \end{array}
    \right)
\end{equation}
  
  \end{proposition}

The above proposition allows us to glean all information from $\Mod(\g)$ from $\even{\Mod(\g)}$ in the following way.
\begin{corollary}
  If $M$ is a $\g$-module, $\Pi(M)$ is a $\g$-module. Further, an odd homomorphism $M \to N$ is simply an even homomorphism $M \to \Pi(N)$.
\end{corollary}


\section{Universal Enveloping Superalgebras}
\label{sec:universal-enveloping-superalgebras}

When studying representations of an algebraic object $G$, it is useful to find a ring $R$ whose modules correspond precisely to $G$-representations. This section is devoted to constructing the universal enveloping superalgebra $\U_s(\g)$ associated to a Lie superalgebra, such that $\g$-modules 
\begin{definition}
  The \emph{universal enveloping superalgebra} of a Lie superalgebra $\g$ is an associative superalgebra $\U_s(\g)$ equipped with a morphism $i: \g \to \U_s(\g)$ such that given any other Lie superalgebra $\V$ with a morphism $j: \g \to \V$ there exists a unique homomorphism $\theta: \U_s(\g) \to \V$ such that $j = \theta \circ i$.

  Explicitly, a universal enveloping superalgebra may be obtained as a quotient of the tensor superalgebra\footnote{Simply the tensor algebra, with grading remembered.} by the ideal generated by elements of the form $[x,y] - x \otimes y (-1)^{\bar x \bar y} y \otimes x$.
\end{definition}

\begin{proposition}
  The following categories are equivalent:
  \begin{enumerate}
  \item The category of graded $\U_s(\g)$-modules in the sense of ring theory.
  \item The category of $\g$-modules in the sense of Section \ref{sec:modules}.
\end{enumerate}
\end{proposition}

%%% Local Variables:
%%% TeX-master: "../dissertation.tex"
%%% End:
\chapter{Relative Cohomology of Lie Superalgebras}
\label{chap:rel-cohom}
\section{Overview}
\label{sec:coh-overview}

Relative cohomology of Lie superalgebras generalizes the cohomology theory of Lie algebras in two ways. When both generalizations are utilized simultaneously, geometrically meaningful cohomology rings arise. This is in stark contrast to ordinary Lie algebra cohomology rings, which have Krull dimension zero and are indeed finite-dimensional vector spaces.

The first generalization is to consider Lie superalgebras rather than Lie algebras. The Koszul complex used to compute Lie superalgebra cohomology is nonzero in infinitely many degrees, potentially leading to cohomology rings of positive Krull dimension. Unfortunately, it was proved by Fuks-Leites that this is rarely the case \cite{fuks-leites}.

The second generalization is to consider cohomology \emph{relative} to a subsuperalgebra. Remarkably, in Lie superalgebra theory, relative cohomology often times yields cohomology groups that are larger than their absolute counterparts. Relative cohomology of Lie algebras was first considered by Fuks \cite{fuks}, and fits into the relative cohomology theory of Hochschild \cite{hochschild}.

\section{Koszul Complex}
\label{sec:koszul}

  Let $\g$ be a Lie superalgebra, $\a \leq \g$ a subsuperalgebra, and $M$ a $\g$-supermodule. The \emph{$p^\text{th}$ cochain of $(\g,\a)$ with coefficients in $M$} is the $\a$-module
  \[
    \C^p(\g,\a;M) = \Hom_\a\left( \superext{p}(\g/\a), M \right)
  \]
  The \emph{coboundary map} $d: \C^p(\g,\a;M) \to \C^{p+1}(\g,\a;M)$ is defined by
\begin{align*}
  df(\w_0 \wedge \ldots \wedge \w_n) = &\sum_{i = 0}^n (-1)^{\tau_i(\bar \w_0, \ldots,\bar\w_n,\bar f)} \w_i .f(\w_0 \wedge \ldots \hat \w_i \ldots \wedge \w_n) \\
 &+ \sum_{i < j} (-1)^{\sigma_{i,j}(\bar \w_0, \ldots , \bar \w_n)} f([\w_i,\w_j] \wedge \w_0 \wedge \ldots \hat \w_i \ldots \hat \w_j \ldots \wedge \w_n)
\end{align*}
where parities $\tau_i$ and $\sigma_{i,j}$ follow the formulae
\begin{align*}
  \tau_i(\alpha_0,\ldots,\alpha_n,\beta) &= i + \alpha_i (\alpha_0 + \ldots + \alpha_{i-1} + \beta)\\
  \sigma_{i,j}(\alpha_0,\ldots,\alpha_n) &= i + j + \alpha_i \alpha_j + \alpha_i(\alpha_0 + \ldots + \alpha_{i-1}) + \alpha_j (\alpha_0 + \ldots + \alpha_{j-1})
\end{align*}

Composing these maps yields a diagram:
\begin{equation}\label{eq:complex}
  \ldots \xrightarrow{d} \C^{p-1}(\g,\a;M) \xrightarrow{d} \C^p(\g,\a;M) \xrightarrow{d} \C^{p+1}(\g,\a;M) \xrightarrow{d} \ldots
\end{equation}

\begin{proposition}
  Let $\g$ be a Lie superalgebra, $\a \leq \g$ a submodule, and $M$ a $\g$-module. The morphism
  \[
    d \circ d : \H^{p-1}(\g,\a;M) \to \H^{p+1}(\g,\a;M)
  \]
  is equal to zero. In other words, Equation \ref{eq:complex} is a complex.
\end{proposition}

\begin{definition}
  Let $\g$ be a Lie superalgebra, $\a \leq \g$ a subsuperalgebra, and $M$ a $\g$-supermodule. The \emph{$p^\text{th}$ cohomology group of $(\g,\a)$ with coefficients in $M$} is the $\a$-module
  \[
    \H^p(\g,\a;M) = \frac{\ker\left(d:\C^p(\g,\a;M) \to \C^{p+1}(\g,\a;M)\right)}{\im \left(d:\C^{p-1}(\g,\a;M) \to \C^{p}(\g,\a;M)\right)}
  \]
  
\end{definition}

\section{Products on Cochains and Cohomology}
\label{sec:products}

Consider modules $M_1$, $M_2$, and $N$, with a pairing, i.e., a map of $\g$-modules $m:M_1 \otimes M_2 \to N$. Cochains may be paired
\[
  C^p(\g,\a;M_1) \otimes C^q(\g,\a;M_2) \to C^{p+q}(\g,\a;N)
\]
by making use of the super anaologue of ordinary Grassmann multiplication $\mu: \superext{p}(\g/\a) \otimes \superext{q}(\g/\a) \to \superext{p+q}(\g/\a)$ as follows:
\begin{align}
  \begin{split}
    C^p(\g,\a;M_1) \otimes C^q(\g,\a;M_2) &\cong \Hom_\a\left(\superext{p}(\g/\a),M_1\right) \otimes \Hom_\a\left(\superext{q}(\g,\a),M_2\right) \\
    & \to \Hom_\a\left( \superext{p}(\g/\a) \otimes \superext{q}(\g/\a), M_1 \otimes M_2 \right) \\
    & \to \Hom_\a\left( \superext{p+q}(\g/\a), N\right) \\
    &= C^{p+q}(\g,\a;N)
  \end{split}
\end{align}

We will be most interest in the case $M_1 = M_2 = N = \CC$ and $\CC \otimes \CC \to \CC$ is ordinary multiplication. Of secondary interest is the case when $M_1 = M^*$, $M_2 = M$ and $N = M* \otimes M$ with pairing given by the natural action $\gamma \otimes x \mapsto \gamma(x)$.

This pairing of cochains descends to a well-defined pairing of cohomology groups
\begin{equation}
  \label{eq:cohom-prod}
  \H^p(\g,\a;M_1) \otimes \H^q(\g,\a;M_2) \to H^{p+q}(\g,\a;N)
\end{equation}
which leads to the following definition and theorem.

\begin{definition}
  The \emph{cohomology ring} of $\g$ relative to $\a$ is the set
  \[
    \H^\bullet(\g,\a;\CC) = \bigoplus_{p \geq 0} \H^p(\g,\a;\CC)
  \]
  
\end{definition}

\begin{theorem}
  Let $\g$ be a Lie superalgebra and $\a \leq \g$ a subsuperalgebra. The morphism of Equation \ref{eq:cohom-prod}
\[
  \H^p(\g,\a;M_1) \otimes \H^q(\g,\a;M_2) \to \H^{p+q}(\g,\a;N)
\]
defines a ring structure on $\H^\bullet(\g,\a;M)$. Furthermore, this ring is graded-commutative, meaning that for homogeneous elements $\alpha, \beta \in \H^\bullet(\g,\a;\CC)$, $\alpha \cdot \beta = (-1)^{\bar \alpha \cdot \bar \beta} \beta \cdot \alpha$.
\end{theorem}

\section{Classical Results}
\label{sec:bkn-fuks-leites}

This section is devoted to presenting two theorems which describe the bahavior of relative cohomology at extreme values of $\a \leq \even{\g}$. Namely, the result of Fuks-Leites states that cohomology relative to $\a = 0$ contains very little geometric information. In other words, the cohomology ring is a finite-dimensional vector space. The result of Boe-Kujawa-Nakano states that cohomology relative to $\a = \even{\g}$ carries geometric information and the behavior of this cohomology ring is governed by invariant theory.

\begin{theorem}[Fuks-Leites, {\cite[\S 2.6]{fuks}}]
There are ring isomorphisms relating Lie superalgebra cohomology to Lie algebra cohomology, from which it follows that the Lie superalgebra cohomology is finite-dimensional as a vector space.
  \begin{enumerate}[(a)]
  \item \[\H^\bullet(\gl(m|n),0;\CC) \cong \H^\bullet(\gl(\max(m,n)),0;\CC)\]
  \item \[
            \H^\bullet(\mathfrak{osp}(m|2n),0;\CC) \cong
      \begin{cases}
        \H^\bullet(\mathfrak{o}(m),0;\CC) \text{ if } m \geq 2n \\
        \H^\bullet(\mathfrak{sp}(2n),0;\CC) \text{ if } m < 2n
\end{cases}
    \]
  \end{enumerate}
   A similar statement holds for Lie superalgebras of type $G(3)$, $F(4)$, and $D(2,1;\alpha)$
\end{theorem}

\begin{theorem}[Boe-Kujawa-Nakano, \cite{BKN-1}] \label{thm:bkn}
  Let $\g$ be a classical Lie superalgebra with $\even{\g} = \Lie(\even{G})$. The cohomology ring relative to $\even{\g}$ may be identified as the invariants of the action of $\even{G}$ on polynomials on $\odd{\g}$:
  \begin{equation}
    \label{eq:bkn-thm}
    \H^\bullet(\g,\even{\g};\CC) \cong S^\bullet(\odd{\g}^*)^{\even{G}}.
  \end{equation}
  Because $\even{G}$ is a reductive algebraic group, this cohomology ring is finitely-generated over $\CC$.
\end{theorem}

This essentially follows by looking at the coboundary definition of Section \ref{sec:koszul}, and realizing that all coboundaries disappear when $\a = \even{\g}$. Therefore, the cochain groups are isomorphic to the cohomology groups.

\section{Sequential realization of $\Ext_{(\g,\a)}^n(M,N)$}
\label{sec:ext-realization}

For $\g$ a Lie superalgebra and $\a \leq \g$, the $\Ext$-group of a pair of modules $\Ext_{(\g,\a)}^n(M,N)$ has a handy realization. As in the finite groups case  as the set of all $n$-fold extensions
\begin{equation}
  \label{eq:ext-seq}
  0 \to N \to E_1 \to E_2 \to \ldots \to E_n \to M \to 0
\end{equation}
which split upon restriction to $\a$. 

\section{Computations}
\label{sec:compute-H*}

In this section we present computations of relative cohomology for Lie superalgebras of the form $\left< x\right>$, which were classified in Example \ref{ex:q1}. For more computations, we direct the reader to \cite[Table 1]{BKN-1}.

\begin{example}
  \begin{enumerate}
  \item If $x \in \even{\g}$, then $\left<x\right> \cong \CC$ and $\H^\bullet(\left<x\right>, \even{\left<x\right>} ; \CC) = \CC$.
  \item If $x \in \odd{\g}$ and $[x,x] =0$, then $\g \cong 0 \oplus \CC$. Thus $\H^\bullet(\left<x\right>,\even{\left<x\right>};CC) = \H^\bullet(\left<x\right>,0;\CC)$, and by Theorem \ref{thm:bkn}
    \begin{equation}
      \label{eq:odd-ab-cohom}
      \H^\bullet(\left<x\right>,0;\CC) \cong S(\odd{\g}^*)^{1} \cong \CC[y]
    \end{equation}
  \item If $x \in \odd{\g}$ and $[x,x] \neq 0$, then $\left<x\right>$ is of type $\q(1)$. The adjoint action of $\even{\g} \actson \odd{\g}$ is trivial, so again we use Theorem \ref{thm:bkn} to conclude
    \begin{equation}
      \label{eq:q(1)-cohom}
      \H^\bullet(\q(1),\even{\q(1)};\CC) \cong S(\odd{\g}^*)^{\GG_m} \cong \CC[y]
    \end{equation}
  \end{enumerate}
\end{example}

%%% Local Variables:
%%% TeX-master: "../dissertation.tex"
%%% End:


\chapter{Finite Generation of Relative Cohomology}
\label{ch:finite-generation}


\section{Motivation}

In this chapter, we will prove the following theorem.

\begin{theorem}
  \label{thm:fg}
  Let $\g = \even{\g} \oplus \odd{\g}$ be a classical Lie superalgebra, and $\a \leq \even{\g}$ an (even) subalgebra, and $M$ a $\g$-module.
  \begin{enumerate}[\indent\rm (a)]
    \item There is a spectral sequence $\{E_r^{p,q}\}$ which computes cohomology and satisfies
  \[
    E_2^{p,q}(M) \cong \H^p(\g,\even{\g};M) \otimes \H^q(\even{\g},\a;\CC) \Rightarrow \H^{p+q}(\g,\a;M)
  \]
  For $1 \leq r \leq \infty$, $E_r^{\bullet,\bullet}(M)$ is a module for $E_2^{\bullet,\bullet}(\CC)$. When $M$ is finite-dimensional, $E_2^{\bullet,\bullet}(M)$ is a Noetherian $E_2^{\bullet,\bullet}(\CC)$-module.
    \item Moreover, the cohomology ring $\H^\bullet(\g,\a;\CC)$ is a finitely-generated $\CC$-algebra.
  \end{enumerate}
\end{theorem}

\section{Filtration on Cochains}
\label{sec:filt-cochains}

Let $\g$ be a classical Lie superalgebra and $\a \leq \g$ any Lie subsuperalgebra. Recall the cochains are defined by
\[
    C^n(\g,\a;M) = \Hom_\a \left( \superext{n}(\g/\a),M \right)
\]
Because $\a \leq \g$ is a subsuperalgebra, the equality
\[
  \g/\a \cong \even{\g} / \even{\a} \oplus \odd{\g}/\odd{\a}
\]
holds, allowing the cochains to be decomposed (as $\a$-modules) as follows.
\begin{align} \begin{split} \label{eq:decompose-cochains}
  C^n(\g,\a;M)
  &= \Hom_\a \left( \superext{n}(\g/\a),M \right)\\
  &= \Hom_\a \left( \superext{n}\left(\even{\g}/\even{\a} \oplus \odd{\g}/\odd{\a}\right),M \right)\\
  &= \Hom_\a \left( \bigoplus_{i+j = n} \superext{i}\left(\even{\g}/\even{\a}\right) \otimes \superext{j}\left(\odd{\g}/\odd{\a}\right), M \right) \\
  &= \bigoplus_{i+j=n} \Hom_\a\left( \ext{i}\left(\even{\g}/\even{\a}\right) \otimes S^j\left(\odd{\g}/\odd{\a}\right), M  \right)\\
  &= \bigoplus_{i+j=n} \Hom_\a\left( \ext{i}\left(\even{\g}/\even{\a}\right), S^{j}\left(\odd{\g}/\odd{\a}\right) \otimes M  \right) \\
  &= \bigoplus_{i+j=n} C^i\left(\even{\g},\even{\a}; S^j(\odd{\g}/\odd{\a})\otimes M\right)
\end{split} \end{align}
Equation \ref{eq:decompose-cochains} expresses arbitrary superalternating functions as sums of superalternating functions with $i$ arguments coming from $\even{\g}/\even{\a}$, and $j$ arguments coming from $\odd{\g}/\odd{\a}$.

Our filtration is inspired by that of \cite{HS-53}, and corresponds to limiting the number of arguments that may come from $\even{\g}/\even{\a}$. Explicitly, define
\begin{equation}
  \label{eq:define-filtration}
  C^n(\g,\a;M)_{(p)} = \bigoplus_{\substack{i+j=n \\ i \leq n-p}} C^i\left(\even{\g},\even{\a}; S^j(\odd{\g}/\odd{\a})\otimes M\right)
\end{equation}
This defines a descending filtration
\begin{align}\begin{split}\label{eq:desc-filt}
    C^n(\g,\a;M) = C^n(\g,\a;M)_{(0)} &\supseteq  C^n(\g,\a;M)_{(1)} \supseteq \ldots \\
    &\ldots \supseteq C^n(\g,\a;M)_{(n)}  \supseteq C^n(\g,\a;M)_{(n+1)} = 0
 \end{split}
\end{align}

This filtration satisfies some basic desired properties.

\begin{proposition} \label{prop:grading-properties}
  Let $\g = \even{\g} \oplus \odd{\g}$ be a Lie superalgebra, $\a \leq \even{\g}$ an even subalgebra, $M$ a $\g$-module, and $C^n(\g,\a;M)_{(p)}$ the filtration defined in Equation \ref{eq:define-filtration}.
  \begin{enumerate}[\indent\rm (a)]
  \item This grading respects the differential, i.e., $d(C^n(\g,\a;M)_{(p)}) \subseteq C^{n+1}(\g,\a;M)_{(p)}$, and thus $C^\bullet(\g,\a;M)_{(p)}$ is a subcomplex of $C^\bullet(\g,\a;M)$ for all $p$.
  \item $C^n(\g,\a;M)_{(p)}$ is an $\a$-submodule of $C^n(\g,\a;M)$, so $C^\bullet(\g,\a;M)_{(p)}$ is a subcomplex of $\a$-modules.
  \item The filtration is exhaustive, i.e., $C^\bullet(\g,\a;M)_{(0)} = C^\bullet(\g,\a;M)$ and $\bigcap_{p\geq 0} C^\bullet(\g,\a;M)_{(p)} = 0$.
  \end{enumerate}
\end{proposition}
\begin{proof}
  \begin{enumerate}[(a)]
  \item Let $f \in C^n(\g,\a;M)_{(p)}$. This means $f$ vanishes when more than $n-p$ arguments belong to $\even{\g}/\a$. We wish to show that $df \in C^{n+1}(\g,\a;M)_{(p)}$, i.e., that $df$ vanishes when more than $n-p+1$ arguments belong to $\even{\g}/\a$. Let $\alpha_0, \ldots , \alpha_{n-p+1} \in \even{\g}/\a$, while $\beta_{n-p+2},\ldots,\beta_n \in \odd{\g}$. Plugging these into the coboundary formula
    \begin{align*}
      df(\alpha_0 \wedge \ldots \wedge \beta_n)
      &= \sum_{0 \leq i \leq n-p+1} (-1)^{\tau_i(-)} \alpha_i . f(\alpha_0 \wedge \ldots \hat \alpha_i \ldots \wedge \beta_n) \\
      & + \sum_{n-p+2 \leq i \leq n} (-1)^{\tau_i(-)} \beta_i .f(\alpha_0 \wedge \ldots \hat \beta_i \ldots \wedge \beta_n) \\
      &+ \sum_{0 \leq i < j \leq n-p+1 } (-1)^{\sigma_{i,j}(-)} f([\alpha_i,\alpha_j] \wedge \alpha_0 \ldots \hat \alpha_i \ldots \hat \alpha_j \ldots \wedge \beta_n) \\
      & + \sum_{\substack{0 \leq i \leq n-p+1 \\ n-p+2 \leq j \leq n}} (-1)^{\sigma_{i,j}(-)} f([\alpha_i,\beta_j]\wedge \alpha_0 \ldots \hat \alpha_i \ldots \hat \beta_j \ldots \wedge \beta_n) \\
      &+ \sum_{n-p+2 \leq i < j \leq n} (-1)^{\sigma_{i,j}(-)} f([\beta_i,\beta_j] \wedge \alpha_0 \ldots \hat \beta_i \ldots \hat \beta_j \ldots \wedge \beta_n)
    \end{align*}
    Looking at each line of the previous equation, notice that $f$ takes in, respectively, $n-p+1$, $n-p+2$, $n-p+1$, $n-p+2$, and $n-p+3$ arguments lying in $\even{\g}/\a$. Thus each term in each summation individually vanishes. Thus we conclude $df \in C^{n+1}(\g,\a;M)_{(p)}$.
  \item Let $x \in \a$, $f \in C^n(\g,\a;M)_{(p)}$. Thus $f(\w_0 \wedge \ldots \w_{n-1})$ vanishes when $n-p+1$ of the $\w_i$ belong to $\a$. writing out the definition of $(x.f)(\w_0\wedge\ldots\wedge\w_{n-1})$ we realize that each term vanishes when $n-p+1$ of the $\w_i$ belong to $\a$, and thus $x.f \in C^n(\g,\a;M)_{(p)}$.
  \item This follows from writing out the definitions and noting $C^n(\g,\a;M)_{(n+1)} = 0$
  \end{enumerate}
\end{proof}
Because of the properties established in Proposition \ref{prop:grading-properties}, 
\begin{equation}
  \label{eq:spectral-sequence}
  E_{r}^{p,q} \Rightarrow \H\left(C^\bullet(\g,\a;M)\right) = \H^\bullet(\g,\a;\CC)
\end{equation}
which computes cohomology.
\TODO{Decide}{Do I need a chapter on spectral sequences?}

\section{Pages of the Spectral Sequence}
\label{sec:pages}

This section is devoted to investigating the pages of the spectral sequence defined by Equation \ref{eq:spectral-sequence}, and the necessary information is summarized in the following lemma.

\begin{proposition} \label{prop:pages}
  The first three pages of the spectral sequence associated to the filtration of Equation \ref{eq:spectral-sequence} may be identified as follows.
  \begin{enumerate}[\indent\rm (a)]
    \itemsep.5em
  \item $E_0^{p,q} \cong C^q\left(\even{\g},\a;\Hom_\CC \left(\superext{p}(\g/\even{\g}),M\right)\right)$,
  \item $E_1^{p,q} \cong \H^q\left(\even{\g},\a;\Hom_\CC\left(\superext{p}(\g/\even{\g}),M\right)\right)$,
  \item $E_2^{p,q} \cong \H^p(\g,\even{\g};M) \otimes \H^q(\even{\g},\a;\CC)$.
  \end{enumerate}
\end{proposition}

The proof of Proposition \ref{prop:pages} requires the following lemma.

\begin{lemma} \label{lemma:vanishing}
  Let $\even{\g}$ be a reductive Lie algebra, $M$ be a finite-dimensional semisimple $\even{\g}$-module such that $M^{\even{\g}} = 0$. Then $\H^n(\even{\g},\a;M) = 0$ for all $n \geq 0$.
\end{lemma}
\begin{proof}[Proof of Lemma \ref{lemma:vanishing}]
  Suppose $M$ is simple and that $n \geq 0$. The group $Z^n(\even{\g},\a;M) \subseteq C^n(\even{\g},\a;M) \subseteq C^n(\even{\g};M)$ is semisimple. The group $d(C^{n-1}(\even{\g},\a;M))$ is a submodule of $Z^n(\even{\g},\a;M)$, and as such there exists a $\even{\g}$-module complement $V$ so that $Z^n(\even{\g},\a;M) = d(C^{n-1}(\even{\g},\a;M)) \oplus V$. We notice that $\even{\g} . Z^n(\even{\g},\a;M) \subseteq d(C^{n-1}(\even{\g},\a;M))$, meaning $\even{\g} . V = 0$. Thus it suffices to show that every cocycle which is annihilated by $\even{\g}$ is a coboundary.

  Since $\even{\g}$ is reductive we may write $\even{\g} = [\even{\g},\even{\g}] \oplus \z$ where $\z$ denotes the center of $\even{\g}$. Since $M$ is simple either $\z . M = 0$ or no non-zero element of $M$ is annihilated by $Z$. Let $f$ be a cocycle which is annihilated by $\even{\g}$, let $z \in \z$, and let $\w_1,\ldots,\w_n \in \even{\g}$. Then $0 = (z.f)(\w_1 \wedge \ldots \wedge \w_n) = z. f(\w_1 \wedge \ldots \wedge \w_n)$. Thus if $\z.M \neq 0$, it follows that $f = 0$. Now we may suppose $\z.M = 0$ and $M \neq 0$.

  Let $C$ be the annihilator in $\even{\g}$ of $M$, so that $C \supseteq Z$. Since the invariant submodule $M^\even{\g} = 0$, it must be the case $C \neq \even{\g}$. Now $C \cap [\even{\g},\even{\g}]$ is an ideal in the semisimple Lie algebra $[\even{\g},\even{\g}]$, meaning there must be a complementary ideal $S$. Of course, $S$ is a non-zero semisimple ideal of $\even{\g}$, which may be decomposed as $\even{\g} = S \oplus C$. Now $M\res_s$ is simple and the representation of $S$ is one-to-one. Thus the Casimir operator of this representation, $\Gamma$, is an automorphism of $M$ which commutes with all $\even{\g}$-operators on $M$. Furthermore, since $[S,C] = 0$, it is seen that for any relative cocycle $f$, $\Gamma \circ f = dg$ is a coboundary. Hence $f = \Gamma^{-1} \circ dg = d( \Gamma^{-1} \circ g)$ as desired.
\end{proof}

With Lemma \ref{lemma:vanishing} established, we are now ready to prove Proposition \ref{prop:pages}.

\begin{proof}[Proof of Proposition \ref{prop:pages}]
    We proceed in steps, identifying the pages in sequence.
  \begin{enumerate}[(a)]
\item   By definition, $E_0^{p,q} = C^{p+q}(\g,\a;M)_{(p)}/C^{p+q}(\g,\a;M)_{(p+1)}$. Using the direct sum decomposition of Equation \ref{eq:define-filtration}, this is exactly $C^q\left(\even{\g},\a;\Hom_\CC\left(\superext{p}(\g/\even{\g}),M\right)\right)$.
\item Functoriality of the isomorphism of (a), i.e., $E_0^{p,\bullet} \cong C^\bullet(\even{\g},\a;\Hom_\CC(S^p(\g/\even{\g}),M))$ as complexes will imply their cohomologies are equal, i.e.,  $E_1^{p,q} \cong \H^q(\even{\g},\a;\Hom_\CC(S^p(\g/\even{\g}),M))$.

  To deduce functoriality of the isomorphism it will suffice to chase the following diagram.
    \[
    \begin{tikzcd}
      C^{p+q}(\g,\a;M)_{(p)} \arrow[r,"d_{(\g,\a)}"] \arrow[d,twoheadrightarrow] & C^{p+q+1}(\g,\a;M)_{(p)} \arrow[d,twoheadrightarrow] \arrow[dd,bend left=35,"\pi"] \\
      E_0^{p,q} \arrow[r,"d_0"] \arrow[d,"\cong"] & E_0^{p,q+1} \arrow[d,"\cong"] \\
      C^q(\even{\g},\a;\Hom_\CC(S^p(\odd{\g}),M)) \arrow[r,"d_{(\even{\g},\a)}"] \arrow[uu,bend left=35,"i"] & C^{q+1}(\even{\g},\a;\Hom_\CC(S^p(\odd{\g}),M))
    \end{tikzcd}
  \]
  With section $i$ corresponding to the direct sum decomposition given in Equation \ref{eq:decompose-cochains}. The goal is to show the composition $ \pi \circ d_{(\g,\a)}\circ i = d_{(\even{\g},\a)}$. Since $d_0$ is defined by $d_{(\g,\a)}$, this will show the bottom square commutes, resulting in an isomorphism of complexes.

  Choose $f \in C^q(\even{\g},\a;\Hom_\CC(S^p(\odd{\g}),M))$, and notice that $df$ is given by usual Lie algebra differential
  \[
df(\w_0 \wedge \ldots \w_q) = \sum_{i = 0}^q (-1)^i \w_i.f(\w_0 \wedge \ldots \hat \w_i \ldots \wedge \w_q) + \sum_{i < j} (-1)^{i+j}f([\w_i,\w_j] \wedge \w_0 \wedge \ldots \hat \w_i \ldots \hat \w_j \ldots \wedge \w_q)
  \]
  Set $\tilde f = i(f) \in C^{p+q}(\g,\a;M)$. The differential is given by the Lie superalgebra cohomology differential, and we arrive at a formula for $d_{(\g,\a)}f(\w_0 \wedge \ldots \wedge \w_{p+q})$. However, because we are taking a quotient $\pi$, it only matters how $d_{(\g,\a)}\tilde{f}$ behaves with $q+1$ even arguments and $p$ odd arguments. Thus we investigate
  \begin{align*}
    d_{(\g,\a)}f(\alpha_0\wedge \ldots \wedge \alpha_q \wedge \beta_1 \wedge \ldots \wedge \beta_p) &= \sum_{i = 0}^q (-1)^{\tau_i(-)} \alpha_i.\tilde{f}(\alpha_0\wedge \ldots \hat \alpha_0 \ldots \wedge \alpha_q \wedge \beta_1 \wedge \ldots \wedge \beta_p) \\
                                                                                                    &+ \sum_{i = q+1}^{p+q} (-1)^{\tau_i(-)} \beta_{i - q}.f(\alpha_0 \wedge \ldots \wedge \alpha_q \wedge \beta_1 \wedge \ldots \hat \beta_{i-q} \ldots \wedge \beta_p) \\
                                                                                                    &+ \sum_{0 \leq i < j \leq q} (-1)^{\sigma_{i,j}(-)}f([\alpha_i,\alpha_j] \wedge \alpha_0 \ldots \hat \alpha_i \ldots \hat \alpha_j \ldots \beta_p) \\
                                                                                                    &+ \sum_{\substack{0 \leq i \leq q \\ q+1 \leq j \leq p+q}} (-1)^{\sigma_{i,j}(-)} f([\alpha_i,\beta_{j-q}] \wedge \alpha_0 \ldots \hat \alpha_i \ldots \hat \beta_{j-q} \ldots \beta_p) \\
    &+ \sum_{q+1 \leq i < j \leq p+q} (-1)^{\sigma_{i,j}(-)} f([\beta_{i-q},\beta_{j-q}]\wedge \alpha_0 \ldots \hat \beta_i \ldots \hat \beta_j \ldots \wedge \beta_p)
  \end{align*}
  By construction, $\tilde{f}$ vanishes unless exactly $q$ arguments are even and $p$ arguments are odd. This only occurs in the first, third, and fourth lines of the preceding sum. Working out the relevant signs yields
\[
  \tau_i(\underbrace{\bar 0,\ldots,\bar 0}_{q+1},\underbrace{\bar 1,\ldots,\bar 1}_p,\bar f) = i \text{ when } i \leq q
\]
\[
    \sigma_{i,j}(\underbrace{\bar 0,\ldots,\bar 0}_{q+1},\underbrace{\bar 1,\ldots,\bar 1}_p) = \begin{cases}
      i + j &\text{ if } i,j \leq q \\
      i - q - 1 &\text{ if } i \leq q, j \geq q+1
    \end{cases}
 \]
So the previous equation for $d_{(\g,\a)}\tilde f$ becomes
  \begin{align*}
    d_{(\g,\a)}f(\alpha_0\wedge \ldots \wedge \alpha_q \wedge \beta_1 \wedge \ldots \wedge \beta_p) &= \sum_{i = 0}^q (-1)^{i} \alpha_i.\tilde{f}(\alpha_0\wedge \ldots \hat \alpha_0 \ldots \wedge \alpha_q \wedge \beta_1 \wedge \ldots \wedge \beta_p) \\
                                                                                                    &+ \sum_{0 \leq i < j \leq q} (-1)^{i+j}f([\alpha_i,\alpha_j] \wedge \alpha_0 \ldots \hat \alpha_i \ldots \hat \alpha_j \ldots \beta_p) \\
                                                                                                    &- \sum_{\substack{0 \leq i \leq q \\ q+1 \leq j \leq p+q}} (-1)^{i} f(\alpha_0 \ldots \hat \alpha_i \ldots \wedge \alpha_q \wedge [\alpha_i,\beta_{j-q}] \wedge \beta_1\ldots \hat \beta_{j-q} \ldots \beta_p) \\
  \end{align*}
  Now if we compute $d_{(\even{\g},\a)}f$, accounting for the action on $\Hom_\CC(S^p(\odd{\g}),M)$, we arrive at the same formula.
\item Notice first that by semisimplicity $\Hom_\CC(S^n(\g/\even{\g}),M) \cong \Hom_{\even{\g}}(S^n(\g/\even{\g}),M) \oplus V$ where $V$ is some complement with $V^{\even{\g}} = 0$. By the lemma,
  \[
    E_1^{p,q} \cong \H^q(\even{\g},\a;\Hom_{\even{\g}}(S^p(\g/\even{\g}),M)) \oplus \H^q(\even{\g},\a;V) = \H^q(\even{\g},\a;\Hom_{\even{\g}}(S^p(\g/\even{\g}),M)).
  \]
  Because $\even{\g}$ acts trivially on $\Hom_{\even{\g}}(S^p(\g/\even{\g}),M)$, we may conclude that $E_1^{p,q} \cong \H^q(\even{\g},\a;\CC) \otimes \Hom_{\even{\g}}(S^p(\g/\even{\g}),M)$. This association is functorial, i.e., induces an isomorphism $E_1^{\bullet,q} \cong \H^q(\even{\g},\a;\CC) \otimes \Hom_\CC(S^\bullet(\g/\even{\g}),M)$ as complexes. Therefore, we may conclude that $E_2^{p,q} \cong \H^q(\even{\g},\a;\CC) \otimes \H^p(\g,\even{\g};M)$.
\end{enumerate}
This completes the proof of Proposition \ref{prop:pages}.
\end{proof}

\section{Proof of Finite Generation}
\label{sec:fg-proof}

Recall the statement of Theorem \ref{thm:fg}:

\TODO{Numbering}{How do I restate a Theorem the right way?}

\begin{theorem}
  Let $\g = \even{\g} \oplus \odd{\g}$ be a classical Lie superalgebra, and $\a \leq \even{\g}$ an (even) subalgebra, and $M$ a $\g$-module.
  \begin{enumerate}[\indent\rm (a)]
    \item There is a spectral sequence $\{E_r^{p,q}\}$ which computes cohomology and satisfies
  \[
    E_2^{p,q}(M) \cong \H^p(\g,\even{\g};M) \otimes \H^q(\even{\g},\a;\CC) \Rightarrow \H^{p+q}(\g,\a;M)
  \]
  For $1 \leq r \leq \infty$, $E_r^{\bullet,\bullet}(M)$ is a module for $E_2^{\bullet,\bullet}(\CC)$. When $M$ is finite-dimensional, $E_2^{\bullet,\bullet}(M)$ is a Noetherian $E_2^{\bullet,\bullet}(\CC)$-module.
    \item Moreover, the cohomology ring $\H^\bullet(\g,\a;\CC)$ is a finitely-generated $\CC$-algebra.
  \end{enumerate}
\end{theorem}

\begin{proof}
  In fact, the all that is left to show is that for $M$ finite-dimensional, $E_2^{\bullet,\bullet}(M)$ is a Noetherian $E_2^{\bullet,\bullet}(\CC)$-module, and that (b) follows from (a). The $E_2^{p,q}(M)$-page identification appears in Proposition \ref{prop:pages} of the previous section.

    As such, let $M$ be a finite-dimensional $\g$-module. $E_2^{\bullet,\bullet}(M)$ is a Noetherian $S^\bullet(\odd{\g}^*)^{\even{G}}$-module via the map
  \[
    S^\bullet(\odd{\g}^*)^{\even{G}} \hookrightarrow E_2^{\bullet,0}(\CC) \subseteq E_2^{\bullet,\bullet}(\CC).
  \]
  $E_\infty^{\bullet,\bullet}(M)$, being a section of $E_2^{\bullet,\bullet}(M)$ is a Noetherian $S^\bullet(\odd{\g}^*)^{\even{G}}$-module via the map
  \[
    S^\bullet(\odd{\g}^*)^{\even{\g}} \to E_\infty^{\bullet,0}(\CC) \subseteq E_\infty^{\bullet,\bullet}(\CC).
  \]
  Consequently, $E_\infty^{\bullet,\bullet}(M)$ is a Noetherian $E_\infty^{\bullet,\bullet}(\CC)$-module.

That the cohomology ring is finitely generated follows from this: Because $\CC$ is a Noetherian $\g$-module,  $E_\infty^{\bullet,\bullet}(\CC)$ is a Noetherian $E_\infty^{\bullet,\bullet}(\CC)$-module, however $E_\infty^{\bullet,\bullet}(\CC) = \Gr(\H^\bullet(\g,\a;\CC))$, the associated graded module of the cohomology ring. Because Noetherian associated graded modules come from Noetherian modules, we may conclude that $\H^\bullet(\g,\a;\CC)$ is a Noetherian $\H^\bullet(\g,\a;\CC)$-module, and the cohomology ring is therefore finitely generated.
\end{proof}

We conclude this section by identifying the edge homomorphsim of the spectral sequence as the natural restriction morphism
\begin{equation}
  \label{eq:restriction}
  \resmap: \H^\bullet(\g,\even{\g};M) \to \H^\bullet(\g,\a;M)
\end{equation}
Which, in the case $M = \CC$, makes $\H^\bullet(\g,\a;\CC)$ into an integral extension of a quotient of $\H^\bullet(\g,\even{g};\CC)$.

\begin{proposition}
  The edge homomorphism of the spectral sequence corresponds to the natural restriction homomorphism of cohomology rings.
\end{proposition}
\begin{proof}
The restriction map $C^n(\g,\even{\g};\CC) \xrightarrow{\resmap} C^n(\g,\a;\CC)$ induces a map on cohomology $\H^n(\g,\even{\g};\CC) \xrightarrow{\resmap^*} \H^n(\g,\a;\CC)$. Because restriction repects the filtration of Section \ref{sec:filt-cochains}, the map $\resmap^*$ will respect the induced filtration on cohomology, i.e., $F^p \H^n(\g,\even{\g};\CC) \xrightarrow{\resmap^*} F^p\H^n(\g,\a;\CC)$. This descends to a map on the associated graded of each cohomology ring, which may be precomposed with the projection onto associated graded as follows
  \[
    \H^n(\g,\even{\g};\CC) \to \Gr\left(\H^n(\g,\even{\g};\CC) \right) \to \Gr\left( \H^n(\g,\a;\CC) \right)
  \]

\end{proof}

\begin{corollary}
  When $\a \leq \even{\g}$,
  \[
    \dim_{Kr}\H^\bullet(\g,\a;\CC) \leq \dim_{Kr}\H^\bullet(\g,\even{\g};\CC)
  \]
  
\end{corollary}
\begin{proof}
  The corollary follows from the fact that an integral extension has Krull dimension no greater than the base, and quotients can have smaller Krull dimension.
\end{proof}



\section{A Cohomology Ring of Intermediate Dimension}
\label{sec:int-dim}

  In many instances, Lie superalgebra cohomology $\H^\bullet(\g;\CC) = \H^\bullet(\g,0;\CC)$ will vanish in all but finitely many degrees (see \cite{fuks-leites} or \cite[Th\'eor\`eme 5.3]{MR1450424}), leading one to conclude the ring has Krull dimension zero and thus uninteresting geometry. Here it is shown that for $\g = \gl(1|1)$ and $\a$ generated by $\operatorname{diag}(1 \mid 1) \in \gl(1|1)$, $\H^\bullet(\g,\a;\CC)$ is nonzero in infinitely many degrees. From this, we may conclude $\H^\bullet(\g,\a;\CC)$ has positive Krull dimension. This is an especially nice case; $\a$ acts trivially on $\gl(1|1)$ so every map $\superext{n}(\g/\a) \to \CC$ is $\a$-invariant.

  Take the basis for $\gl(1|1)/\a$
  \[
    \alpha = \begin{pmatrix} 1 & 0 \\ 0 & 0\end{pmatrix}\text{ , } \beta_1 = \begin{pmatrix}0 & 1 \\ 0 & 0\end{pmatrix} \text{ , } \beta_2 = \begin{pmatrix}0 & 0 \\ 1 & 0\end{pmatrix}
  \]
  $\superext{2n}(\g/\a)$ has basis $\{\alpha \otimes \beta_1^i \beta_2^j\}_{i + j + 1 = n} \cup \{\beta_1^i \beta_2^j\}_{i+j = n}$. Consider $f \in C^{2i}(\g,\a;\CC)$ which maps $\beta_1^n \beta_2^n$ to 1 and all other basis vectors to zero. Since $\CC$ has the trivial action, $df$ has the form
  \[
    df(\w_0 \wedge \ldots \wedge \w_{2n}) = \sum_{i = 0}^p (-1)^{\sigma_{i,j}(\bar \w_0,\ldots,\bar\w_{2n})} f([\w_i,\w_j] \wedge \w_0 \wedge \ldots \hat \w_i \ldots \hat \w_j \ldots \wedge \w_{2n})
  \]
  By inspection, $df$ will vanish on all basis vectors $\beta_1^i \beta_2^j$ and $df(\alpha \otimes \beta_1^i \beta_2^j) = (i-j) f(\beta_1^i \beta_2^j)$. This is $0$ when $i,j \neq n$ by definition of $f$, and when $i = j = n$ this is zero because the coefficient vanishes. So $f$ is a cocycle.

  Suppose $dg = f$ for some $g \in C^{2n-1}(\g,\a;\CC)$. Then we compute $dg(\beta_1^n \beta_2^n)$, which is a sum of terms of the form $(-1)^{\sigma_{i,j}(-)} g([\beta_k, \beta_l] \wedge \beta_1^{n_1} \wedge \beta_2^{n_2}$, each of which vanishes individually so that $dg(\beta_1^n \beta_2^n) = 0$.

  Therefore, $f$ is \emph{not} a coboundary. So for every $n \geq 2$, $\H^{2n}(\g,\a;\CC) \neq 0$. This shows that cohomology relative to an even subalgebra heuristically lies somewhere between the results of Fuks-Leites \cite{fuks-leites} and Boe-Kujawa-Nakano \cite{BKN-1}.

\section{Structure of Cohomology Rings}
\label{sec:structure}

The spectral sequence of Section \ref{sec:filt-cochains} allows us to investigate the properties of cohomology rings in certain cases. There are certain conditions on the spectral sequence that appear quite often and it is shown that these cohomology rings are particularly nicely behaved.

\subsection{Cohen-Macaulay Cohomology Rings}
\label{sec:cm-H*}

The following theorem is motivated by \cite[Proposition 3.1]{MR3233523}. The reader should recall that an algebra $A$ is \emph{Cohen-Macaulay} if there is a polynomial subalgebra over which $A$ is a finite and free module, see \cite[\S 5.4]{MR1634407}.

\begin{proposition}
  \label{prop:collapse-CM}
  Let $\g = \even{\g} \oplus \odd{\g}$ be a classical Lie superalgebra, and $\a \leq \even{\g}$ a subalgebra. If the spectral sequence constructed in Section \ref{sec:spectral-sequence} collapses at $E_2$ (i.e., if $E_2^{\bullet,\bullet}(\CC) \cong E_\infty^{\bullet,\bullet}(\CC)$), then $\H^\bullet(\g,\a;\CC)$ is a Cohen-Macaulay ring.
\end{proposition}
\begin{proof}
  The spectral sequence $E_2^{\bullet,\bullet} = E_\infty^{\bullet,\bullet}$ is a filtered version of the cohomology ring $\H^\bullet(\g,\a;\CC)$. As such, if $\zeta \in E_2^{i,j}$ and $\eta \in E_2^{r,s}$, then $\zeta \cdot \eta \in \sum_{\ell \geq 0} E_2^{i + r + \ell, j + s - \ell}$. Because of this, for any $m \geq 0$, the direct sum of the lowest $m$ rows, denoted $U_m = \sum_{q \leq m} E_2^{\bullet,q}$, is a module for the bottom row $U_0 = E_0^{\bullet,0} \cong \H^0(\even{\g},\a;\CC) \otimes S^\bullet(\odd{\g}^*)^{\even{G}} \cong S^\bullet(\odd{\g}^*)^{\even{G}}$, which by \cite{MR0347810} is a Cohen-Macaulay ring. Because the spectral sequence collapses, $E_2 = E_\infty$ and the quotients $U_m / U_{m-1} \cong \H^m(\even{\g},\a;\CC) \otimes S^\bullet(\odd{\g}^*)^{\even{G}}$ are free $S^\bullet(\odd{\g}^*)^{\even{G}}$-modules. This means the quotient maps $U_m \to U_m / U_{m-1}$ split as maps of $S(\odd{\g}^*)^{\even{G}}$-modules and the proposition follows.
\end{proof}

This example restricts to the case that cohomology of $\g$ relative to $\even{\g}$ vanishes in odd degrees. While this may seem restrictive, \cite[Table 1]{BKN-1} reveals that there are a great many classical Lie superalgebras whose cohomology lives in even degree.

\begin{corollary}
  Let $\g = \even{\g} \oplus \odd{\g}$ be a Lie superalgebra of type $\gl(m|n)$, $\sl(m|n)$, $\mathfrak{psl}(2n|2n)$, $\mathfrak{osp}(2m+1|2n)$, $\mathfrak{osp}(2m|2n)$, $P(4\ell - 1)$, $D(2,1;\alpha)$, $G(3)$, or $F(4)$. Let $\l \leq \even{\g}$ be a standard Levi subalgebra. The following hold:
  \begin{enumerate}[\indent\rm(a)]
  \item $\H^\bullet(\g,\l;\CC)$ is a Cohen-Macaulay ring.
  \item $\krdim \H^\bullet(\g,\l;\CC) = \krdim S^\bullet(\odd{\g}^*)$.
  \end{enumerate}
\end{corollary}


\subsection{Krull Dimensions}
\label{sec:krull-dims}

In this section we present some applications in which we use the spectral sequence of Section \ref{sec:spectral-sequence} to compute Krull dimensions of cohomology rings in particularly nice cases. The reader should notice these results rely on deep results from representation theory in the relative Category $\O$ (cf. \cite[\S 8]{MR2428237}).

\begin{theorem} \label{prop:E2-collapse}
  Let $\g = \even{\g} \oplus \odd{\g}$ be a classical Lie superalgebra such that $S^\bullet(\odd{\g})^{\even{G}}$ vanishes in odd degrees, and $\l \leq \even{\g}$ a standard Levi subalgebra (i.e., nonzero and generated by simple roots). The following hold.
  \begin{enumerate}[\indent\rm (a)]
    \item The spectral sequence of Section \ref{sec:spectral-sequence} collapses at the $E_2$ page and $E_2^{\bullet,\bullet}(\CC) \cong E_\infty^{\bullet,\bullet}(\CC)$.
    \item $\H^\bullet(\g,\l;\CC)$ is Cohen-Macualay,
    \item $\krdim \H^\bullet(\g,\even{\g};\CC) = \krdim \H^\bullet(\g,\l;\CC)$.
  \end{enumerate}
\end{theorem}

\begin{proof}
  We establish (a). Parts (b) and (c) follow by application of Proposition \ref{prop:collapse-CM}.

  Let $\g = \even{\g} \oplus \odd{\g}$ be a classical Lie superalgebra such that $S^\bullet(\odd{\g}^*)^{\even{G}}$ is zero in odd degrees, and $\l \leq \even{\g}$ a Levi subalgebra. According to the Kazhdan-Lusztig conjectures\footnote{When $\h \leq \even{\g}$ is a Cartan subalgebra, $\Ext^n_\O(M,N) \cong \Ext^n_{(\even{\g},\h)}(M,N)$ (see \cite[Theorem 6.15]{MR2428237}). The fact that $\Ext^n_{(\even{\g},\h)}(\CC,\CC)$ vanishes in odd degrees follows from \cite{MR1245719}.}, $\H^\bullet(\even{\g},\l;\CC)$ is only nonzero in even degrees. Section \ref{sec:pages} realizes the $E_2$ page of the Hochschild-Serre spectral sequence as
\[
  E_2^{p,q}(\CC) \cong \H^q(\even{\g},\l;\CC) \otimes S^p(\odd{\g}^*)^{\even{G}}.
\]
Because the differential $d_2:E_2^{p,q} \to E_2^{p+2,q-1}$ descends one row, either $E_2^{p,q} = 0$ or $E_2^{p+2,q-1} = 0$. In either case, $d_2 = 0$ and thus $E_3^{p,q} = E_2^{p,q}$ meaning that $E_3^{p,q}$ vanishes unless $p$ and $q$ are both even. By a similar argument, the differential $d_3:E_3^{p,q} \to E_3^{p+3,q-2}$ must be zero since one of $E_3^{p,q}$ or $E_3^{p+3,q-2}$ will have odd horizontal coordinate and thus be zero. So $E_3^{p,q} \cong E_4^{p,q}$. By induction, this trend continues to arrive at the conclusion that $E_2^{p,q} \cong E_\infty^{p,q}$. This yields the following statement.
\end{proof}

This example restricts to the case that cohomology of $\g$ relative to $\even{\g}$ vanishes in odd degrees. While this may seem restrictive, \cite[Table 1]{BKN-1} reveals that there are a great many classical Lie superalgebras whose cohomology lives in even degree.

\begin{corollary}
  Let $\g = \even{\g} \oplus \odd{\g}$ be a Lie superalgebra of type $\gl(m|n)$, $\sl(m|n)$, $\mathfrak{psl}(2n|2n)$, $\mathfrak{osp}(2m+1|2n)$, $\mathfrak{osp}(2m|2n)$, $P(4\ell - 1)$, $D(2,1;\alpha)$, $G(3)$, or $F(4)$. Let $\l \leq \even{\g}$ be a standard Levi subalgebra. The following hold:
  \begin{enumerate}[\indent\rm(a)]
  \item $\H^\bullet(\g,\l;\CC)$ is a Cohen-Macaulay ring.
  \item $\krdim \H^\bullet(\g,\l;\CC) = \krdim S^\bullet(\odd{\g}^*)$.
  \end{enumerate}
\end{corollary}

%%% Local Variables:
%%% TeX-master: "../dissertation.tex"
%%% End:
\chapter{Support Variety Theory}
\label{ch:support}

\section{Motivation}
\label{sec:supp-motivation}

The finite-generation result of Chapter \ref{ch:finite-generation} opens the door to use the powerful machinery of algebraic geometry when studying cohomology of classical Lie superalgebras relative to an even subsuperalgebra.

In this chapter, the cohomology variety $\V_{(\g,\a)}(\CC)$ is defined, along with support varieties

$\V_{(\g,\a)}(M) \subseteq \V_{(\g,\a)}(\CC)$ for each module $M$. Natural mappings of cohomology rings yield natural mappings of cohomology varieties.

Conjectures are presented, and of particular interest is the elusive tensor-product-theorem, i.e., $\V_{(\g,\a)}(M \otimes N) = \V_{(\g,\a)}(M) \cap \V_{(\g,\a)}(N)$. The tensor product theorem has been established in several contexts in varying levels of generality. The proof always relies on concrete details, primarily through the use of explicit rank varieties. 

\section{Definition and Basic Properties}
\label{sec:def-sv}

Let $\g$ be a classical Lie superalgebra and $\a \leq \even{\g}$ a subsuperalgebra. The cohomology ring $\H^\bullet(\g,\a;\CC)$ is a graded-commutative ring, and as such the subring
\begin{equation} \label{eq:even-cohomology}
  \H^{ev}(\g,\a;\CC) = \bigoplus_{n \in \ZZ_{\geq 0}} \H^{2n}(\g,\a;\CC) \subseteq \H^\bullet(\g,\a;\CC)
\end{equation}
is a commutative, finitely-generated subring of $\H^\bullet(\g,\a;\CC)$, by Theorem [??]. This leads to the first definition of this chapter.


\begin{definition}
  Let $\g = \even{\g} \oplus \odd{\g}$ be a classical Lie superalgebra with $\a \leq \even{\g}$ an even subsuperalgebra. The \emph{cohomology variety of $\g$ relative to $\a$} is the spectrum of the even subring of Equation \ref{eq:even-cohomology}:
  \[
    \V_{(\g,\a)}(\CC) = \maxspec\left( \H^{ev}(\g,\a;\CC)\right)
  \]
\end{definition}

For each $\g$-module $M$, $\Ext_{(\g,\a)}(M,M)$ is a graded module for the cohomology ring $\H^\bullet(\g,\a;\CC) = \Ext_{(\g,\a)}^\bullet(\CC,\CC)$ via the tensor product or cup product, as in Section [??]. Of course, $\Ext_{(\g,\a)}^\bullet(M,M)$ is a graded module for the subring $\H^{ev}(\g,\a;\CC)$. This means the annihilator
\[
  \Ann_{\H^{ev}(\g,\a;\CC)}\left(\Ext_{(\g,\a)}^\bullet(M,M)\right) \trianglelefteq \H^{ev}(\g,\a;\CC)
\]
is a homogeneous ideal for the even-degree subring of the cohomology ring.

\begin{definition}
  Let $\g = \even{\g} \oplus \odd{\g}$ be a classical Lie superalgebra with $\a \leq \even{\g}$ an even subsuperalgebra. The \emph{relative support variety} of $M$ is the vanishing set of its annihilator. In other words,
  \[
    \V_{(\g,\a)}(M) = \Z\left( \Ann_{\H^{ev}(\g,\a;\CC)}\left(\Ext_{(\g,\a)}^\bullet(M,M)\right)\right) \subseteq \V_{(\g,\a)}(\CC)
  \]
\end{definition}

Immediately, we may rephrase common properties of modules in terms of support varieties.

\begin{proposition}
  \begin{enumerate}
  \item For any $\g$-module $M$, $\V_{(\g,\a)}(M)$ is a closed, conical subvariety of $\V_{(\g,\a)}(\CC)$.
  \item For any $\g$-modules $M_1$ and $M_2$, $\V_{(\g,\a)}(M_1 \oplus M_2) = \V_{(\g,\a)}(M_1) \cup \V_{(\g,\a)}(M_2)$.
  \item Whenever $0 \to M_1 \to M_2 \to M_3 \to 0$ is a short exact sequence of $\g$-modules, and $\sigma \in \mathfrak{S}_3$ is a permutation of three letters, $\V_{(\g,\a)}(M_{\sigma(1)}) \subseteq \V_{(\g,\a)}(M_{\sigma(2)}) \cup \V_{(\g,\a)}(M_{\sigma(3)})$.
\end{enumerate}
\end{proposition}

\section{Natural Maps of Cohomology Varieties}
\label{sec:natural-maps}

In this section we exploit the realization of cohomology groups as $n$-fold extensions to see how relations between Lie superalgebras become morphisms of their associated support varieties.

Recall the realization
\begin{equation}
  \label{eq:ses-cohom}
  \H^n(\g,\a;\CC) = \{ 0 \to \CC \to E_1 \to \ldots \to E_n \to \CC \to 0 \mid \circledast \} / \sim
\end{equation}
where $\circledast$ is the condition that the sequence is exact as a sequence of $\g$-modules and splits on restriction to $\a$, and $\sim$ is an equivalence reaction obtained from the pre-equivalence relation of there existing morphisms between extensions.
\TODO{Structure}{Maybe this should go into the section on cohomology rings.}

\begin{definition}
  A \emph{relative subsuperalgebra} is a quadruple $(\b \leq \h, \a \leq \g)$. Here $\h \leq \g$ is a classical subsuperalgebras, in the sense that $\even{\h} \leq \even{\g}$ and $\odd{\h} \leq \odd{\g}$. Further, $\a$ is a subsuperalgebra of $\g$ and $\b$ is a subsuperalgebra of $\h$ which is also contained in $\a$. See Figure \ref{fig:relative-pair} for a pictographic definition.
\end{definition}

\begin{figure}[h]
  \centering
  \begin{tikzcd}
    \a \arrow[r,hook] & \g \\
    \b \arrow[u,hook] \arrow[r,hook] & \h \arrow[u,hook]
  \end{tikzcd}
  \caption{Relative subsuperalgebra}
  \label{fig:relative-pair}
\end{figure}




In the case that $\b \leq \h$ is a relative subsuperalgebra of the pair $\a \leq \g$, there is a natural restriction morphism of cohomology rings:
\begin{equation}
  \label{eq:nat-cohom}
  \resmap: \H^\bullet(\g,\a;\CC) \to \H^\bullet(\h,\b;\CC)
\end{equation}
This yields a natural morphism of cohomology varieties
\begin{equation}
  \label{eq:nat-supp}
  \resmap^*: \V_{(\h,\b)}(\CC) \to \V_{(\g,\a)}(\CC).
\end{equation}

There are several special cases in which the morphism of Equation \ref{eq:nat-supp} is especially useful. By Theorem [??], $\H^\bullet(\g,\a;\CC)$ is an integral extension of a quotient of $\H^\bullet(\g,\even{\g};\CC)$ via the restriction morphism (which by [??] is the edge homomorphism of the spectral sequence). This means that the morphism of varieties
\[
  \resmap^* : \V_{(\g,\a)}(\CC) \to \V_{(\g,\even{\g})}(\CC)
\]
is a finite-to-one map. Further, by the results of Boe-Kujawa-Nakano \cite{BKN-1} the cohomology variety $\V_{(\g,\even{\g})}(\CC)$ may be realized as closed orbits
\[
  \V_{(\g,\even{\g})}(\CC) = \left\{ \even{G}.x \mid x \in \odd{\g} \text{ and } \even{G}.x \text{ is closed} \right\}.
\]
This proves to be an invaluable morphism, allowing us to realize elements of the support variety $\V_{(\g,\a)}(\CC)$ as closed orbits in the space $\V_{(\g,\even{\g})}(\CC)$.

\section{Rank Varieties}
\label{sec:rank-varieties}

While many common properties of support varieties follow from the general theory of modules for rings, one result that requires explicit, context-dependent computations is the proof of the elusive \emph{tensor product theorem}, stated below as a conjecture.

\begin{conjecture}[Tensor Product Theorem]
  \label{conj:tpt}
  Let $\g = \even{\g} \oplus \odd{\g}$ be a classical Lie superalgebra and $\a \leq \even{\g}$ an even subsuperalgebra. Let $M$ and $N$ be two $\g$-modules, then we may identify the support variety of their tensor product as follows:
  \[
    \V_{(\g,\a)}(M \otimes N) = \V_{(\g,\a)}(M) \cap \V_{(\g,\a)}(N).
  \]
\end{conjecture}

In many cases, the path to this theorem depends on the establishment of a rank variety description of the support variety $\V_{(\g,\a)}(M)$.

\begin{definition}
  \label{def:rank-variety}
  Let $\g = \even{\g} \oplus \odd{\g}$ be a classical Lie superalgebra and $\a \leq \even{\g}$ an even subsuperalgebra. The \emph{rank variety} of $\g$ relative to $\a$ is the variety
  \[
    \V_{(\g,\even{\g})}^\#(M) = \left\{ \even{G}.x \mid x \in \odd{\g}, \even{G}.x \text{ is closed, and } M\res_{\left<x\right>}\text{ is not projective}  \right\} \cup \{0\}
  \]
  The study of the structure of $\left<x\right>$ was conducted in Example \ref{ex:q1} and the cohomology rings were identified in Example [??]. 
\end{definition}

\section{}Let $\g$ be a classical Lie superalgebra with $\a \leq \even{\g}$ a subalgebra. We showed in Theorem \ref{thm:spectral-sequence} that $\H^\bullet(\g,\a;\CC)$ is a finitely-generated graded-commutative $\CC$-algebra. Therefore, the subring of even cohomology classes $\H^{ev}(\g,\a;\CC) = \bigoplus \H^{2\bullet}(\g,\a;\CC)$ is commutative and finitely-generated over $\CC$. The \textit{cohomology variety} of $\g$ relative to $\a$ is the algebraic variety
\[
  \V_{(\g,\a)}(\CC) = \maxspec\left(\H^{ev}(\g,\a;\CC)\right).
\]
Note that since $\H^{ev}(\g,\a;\CC)$ is graded we just as well could have looked at the projectivization of $\V_{(\g,\a)}(\CC)$. When dealing with questions of connectivity it will be advantageous to use the projectivization, but in other contexts we will focus exclusively on the conical affine variety.

Recall that $\Ext_{(\g,\a)}^\bullet(M,M)$ is a module over $\H^\bullet(\g,\a;\CC)$, so its annihilator defines a subvariety called the \textit{support variety} of $M$, denoted
\[
  \V_{(\g,\a)}(M) = \Z\left({\Ann_{\H^{ev}(\g,\a;\CC)} \Ext_{(\g,\a)}^\bullet(M,M)}\right) \subseteq \V_{(\g,\a)}(\CC)
\]
where $\Z(I)$ denotes the vanishing set of $I$.

An alternative definition of the support variety is
\[
  \V_{(\g,\a)}(M) = \left\{~ \m \in \V_{(\g,\a)}(\CC) \mid \Ext_{(\g,\a)}^\bullet(M,M)_{\m} \neq 0 ~\right\}.
\]
The following are basic properties whose proof is standard and may be found in \cite{FP-unipotent}.

\begin{enumerate}
  \item For any $\g$-module $M$, $\V_{(\g,\a)}(M)$ is a closed, conical subvariety of $\V_{(\g,\a)}(\CC)$.
  \item For any $\g$-modules $M_1$ and $M_2$, $\V_{(\g,\a)}(M_1 \oplus M_2) = \V_{(\g,\a)}(M_1) \cup \V_{(\g,\a)}(M_2)$.
  \item Whenever $0 \to M_1 \to M_2 \to M_3 \to 0$ is a short exact sequence of $\g$-modules, and $\sigma \in \mathfrak{S}_3$ is a permutation of three letters, $\V_{(\g,\a)}(M_{\sigma(1)}) \subseteq \V_{(\g,\a)}(M_{\sigma(2)}) \cup \V_{(\g,\a)}(M_{\sigma(3)})$.
\end{enumerate}

In this section, we use the realization map $\Phi:\V_{(\g,\a)}(\CC) \to \V_{(\g,\even{\g})}(\CC)$ induced by restriction $\resmap: \H^\bullet(\g,\even{\g};\CC) \to \H^\bullet(\g,\a;\CC)$ to determine properties of (the image of) $\V_{(\g,\a)}(M)$. This has the advantage of taking the elusive, abstract support variety and embedding it inside of something concrete -- indeed, $\V_{(\g,\even{\g})}(\CC)$ is simply the set of closed orbits of the action $\even{G}$ on $\odd{\g}$.

% \section{Carlson supermodules}
% \label{sec:carlson-supermodules}

% Recall that the \emph{syzygy} $\Omega(M)$ of a $(\g,\a)$-module $M$ is defined to be the kernel of a relative-projective cover (in the $(\g,\a)$-category) of $M$. Further syzygies may be defined as $\Omega^0(M) = M$ and $\Omega^{n+1}(M) = \Omega(\Omega^n(M))$. An element $\zeta \in \H^n(\g,\a;M)$, is represented by a morphism $\Omega^n(M) \to \CC$, and its \emph{Carlson supermodule} is defined by $L_\zeta = \ker(\zeta: \Omega^n(M) \to \CC)$.

% In the case that $\a = \even{\g}$, the projectives are finite-dimensional. For arbitrary $\a$, this may not be the case, as the $(\g,\a)$-category may fail to be self-injective. As such, we must be very careful when dealing with (possibly infinite-dimensional) Carlson modules.

% % In the case that $L_\zeta$ is finite-dimensional, we have the following important dictionary between hypersurfaces and modules.

% % \begin{proposition}
% %   Let $\g = \even{\g} \oplus \odd{\g}$ be a Lie superalgebra and $\a \leq \even{\g}$ a subalgebra. Suppose the map $\Phi:\V_{(\g,\a)}(\CC) \to \V_{(\g,\even{\g})}(\CC)$ is a closed embedding. Let $\zeta \in \H^n(\g,\a;\CC)$. Then $\V_{(\g,\a)}(L_\zeta)$ is the hypersurface defined by $\zeta$. We denote this hypersurface $\V_{(\g,\a)}\left<\zeta\right>$.
% % \end{proposition}
% % \begin{proof}
% %   $\Phi(\V_{(\g,\a)}(L_\zeta))$ is a closed subvariety of $\V_{(\g,\even{\g})}(\CC)$, and as such is the intersection of hypersurfaces necessarily of the form $\V_{(\g,\even{\g})}(L_{\hat \zeta_i})$, where $\hat \zeta_i \in \H^{n_i}(\g,\even{\g};\CC)$.
% % \end{proof}

% \begin{proposition}
%   Let $\g = \even{\g} \oplus \odd{\g}$ be a classical Lie superalgebra, $\a \leq \even{\g}$ a subalgebra, and $M$ a finite-dimensional $\g$-module such that $\Omega(M)$ and $\Omega^{-1}(M)$ are both finite-dimensional. The following equalities hold:
%   \[
%     \V_{(\g,\a)}(M) = \V_{(\g,\a)}(M^*) = \V_{(\g,\a)}(\Omega(M)) = \V_{(\g,\a)}(\Omega^{-1}(M)).
%   \]

% \end{proposition}

% \begin{proposition}
%   If $\zeta_1 \in \H^r(\g,\a;\CC)$ and $\zeta_2 \in \H^s(\g,\a;\CC)$, then there is a short exact sequence
%   \[
%     0 \to \Omega^r L_{\zeta_2} \to L_{\zeta_1 \cdot \zeta_2} \oplus P \to L_{\zeta_1} \to 0.
%   \]

% \end{proposition}

% Given an element $\zeta \in \H^n(\g,\even{\g};\CC)$ and its Carlson module $L_\zeta$, there are two natural ways to obtain a $(\g,\a)$-module. Namely, via restriction on the module level $L_\zeta\res_{(\g,\a)}$ or the restriction on cohomology, $L_{\resmap \zeta}$.

% \begin{proposition}
%   Let $\g = \even{\g} \oplus \odd{\g}$ be a classical Lie superalgebra, $\a \leq \even{\g}$ a subalgebra, and $\zeta \in \H^n(\g,\even{\g};\CC)$ a homogeneous element of cohomology. Let $\bar \zeta = \resmap \zeta \in \H^n(\g,\a;\CC)$. Then
%   \[
%     L_\zeta\res_{(\g,\a)} \cong L_{\bar \zeta}.
%   \]
% \end{proposition}
% \begin{proof}
%   Relative projective modules for $(\g,\a)$ are projective modules for $(\g,\even{\g})$, since relative projectives for $\even{\g}$ are simply summands of $U(\g) \otimes_{U(\a)} L \cong U(\g) \otimes_{U(\even{\g})} U(\even{\g}) \otimes_{U(\a)} \otimes L$. As such, we may take relative projective covers and relative syzygies in the $(\g,\a)$-category to obtain $\Omega^n_{(\g,\a)}(\CC) \cong \Omega^n_{(\g,\even{\g})}(\CC)$. From this, it is clear that $L_{\bar \zeta}$ appears by restriction, and thus the isomorphism holds.
% \end{proof}
% % END DELETED PART

\section{Realizability}
\label{sec:realizability}

In this section we address the question of realizability, initially studied by Carlson \cite{MR723070}. As we are using results of Bagci-Kujawa-Nakano \cite{MR2448087}, we need additional assumptions on the Lie superalgebra $\g$, namely we require the superalgebra is \emph{stable} and \emph{polar} in addition to being classical. These assumptions originate in geometric invariant theory, and hold for $\gl(m|n)$ -- see \cite[\S 3.2-3.3]{BKN-1} for a thorough description.

\begin{definition}
  Let $\g = \even{\g} \oplus \odd{\g}$ be a classical, stable, and polar Lie superalgebra with $\a \leq \even{\g}$ a subalgebra. We say $(\g,\even{\g})$-module $M$ is \emph{natural} (with respect to $\a$) if $\V_{(\g,\even{\g})}(M) \cap \Phi\left(\V_{(\g,\a)}(\CC)\right) = \Phi\left(\V_{(\g,\a)}(M)\right)$. The subalgebra $\a$ is \emph{natural} if every $\g$-module is natural with respect to $\a$.
\end{definition}

The paper of Bagci-Kujawa-Nakano \cite[Theorem 8.8.1]{MR2448087} demonstrated that every closed conical subvariety of $\V_{(\g,\even{\g})}(\CC)$ is realized as the support variety of a $(\g,\even{\g})$-module.

\begin{proposition}
  Let $\g = \even{\g} \oplus \odd{\g}$ be a classical, stable, and polar Lie superalgebra with $\a \leq \even{\g}$ a natural subalgebra. Let $X \subseteq \V_{(\g,\a)}(\CC)$ be a closed, conical subvariety. There exists a $(\g,\a)$-module $M$ such that $\Phi\left(\V_{(\g,\a)}(M)\right) = \Phi(X)$.
\end{proposition}
\begin{proof}
  The realization theorem holds for $(\g,\even{\g})$-modules, so choose $M$ such that $\V_{(\g,\even{\g})}(M) = \Phi(X)$. By naturality, $\Phi(\V_{(\g,\a)}(M)) = \Phi(\V_{(\g,\a)}(\CC)) \cap \V_{(\g,\even{\g})}(M) = \Phi(X)$.
\end{proof}

\section{Tensor products}
\label{sec:tensor-products}

A tensor product theorem gives us the ability to geometrically control the support theory of tensor products of modules. Historically, this has been a very elusive property of support varieties, often times requiring support varieties recognized in some other way. For example, in the case of finite groups, the tensor product theorem was not shown until support varieties were determined to be isomorphic to the very concrete rank varieties \cite{MR621284}.

In this section, we circumvent this issue by considering only superalgebras which satisfy the tensor product theorem relative to $(\g,\even{\g})$, and using the realization map to intersect supports of $(\g,\a)$-modules inside $\V_{(\g,\even{\g})}(\CC)$.

\begin{definition}
  Let $\g = \even{\g} \oplus \odd{\g}$ be a Lie superalgebra with subalgebra $\a \leq \even{\g}$. The pair $(\g,\a)$ is said to satisfy the \emph{tensor product theorem} if $\V_{(\g,\a)}(M \otimes N) = \V_{(\g,\a)}(M) \cap \V_{(\g,\a)}(N)$ for all modules $M,N$.
\end{definition}

Lehrer-Nakano-Zhang proved the tensor product theorem hold for the pair $(\gl(m|n),\even{\gl(m|n)})$, \cite[Theorem 5.2.1]{MR2836115}

\begin{proposition}
  Let $\g = \even{\g} \oplus \odd{\g}$ be a Lie superalgebra which satisfies the tensor product theorem relative to $\even{\g}$, and $\a \leq \even{\g}$ a natural subalgebra of $\g$.  Then $\Phi(\V_{(\g,\a)}(M \otimes N)) = \Phi(\V_{(\g,\a)}(M)) \cap \Phi(\V_{(\g,\a)}(N))$.
\end{proposition}
\begin{proof}
  One has:
  \begin{align*}
    \Phi(\V_{(\g,\a)}(M \otimes N)) &= \Phi(\V_{(\g,\a)}(\CC)) \cap \V_{(\g,\even{\g})}(M \otimes N) \\
                                   &= \left(\Phi(\V_{(\g,\a)}(\CC)) \cap \V_{(\g,\even{\g})}(M)\right) \cap \left( \Phi(\V_{(\g,\a)}(\CC)) \cap \V_{(\g,\even{\g})}(N)\right) \\
    &= \Phi(\V_{(\g,\a)}(M)) \cap \Phi(\V_{(\g,\a)}(N)).
  \end{align*}
\end{proof}

% BEGIN DELETED PART
\begin{proposition}
  Let $\g = \even{\g} \oplus \odd{\g}$ be a classical Lie superalgebra with $\a \leq \even{\g}$ a submodule. Denote by $\Phi: \V_{(\g,\a)}(\CC) \to \V_{(\g,\even{\g})}(\CC)$ the restriction morphism. If $M$ is a $(\g,\even{\g})$-module, then
  \[
    \V_{(\g,\even{\g})}(M) \cap \Phi\left(\V_{(\g,\a)}(\CC)\right) = \Phi\left(\V_{(\g,\a)}(M)\right).
  \]

\end{proposition}
\begin{proof}
  First, suppose $\m \in \Phi\left(\V_{(\g,\a)}(M)\right)$. This means there is a maximal ideal $\widetilde \m \trianglelefteq \H^\bullet(\g,\a;\CC)$ such that  $\resmap^{-1}(\widetilde \m) = \m$ and $\widetilde{\m} \supseteq \Ann_{\H^\bullet(\g,\a;\CC)}\Ext_{(\g,\a)}^\bullet(M,M)$. Of course, $\m \in \Phi\left(\V_{(\g,\even{\g};\CC)}(\CC)\right)$, so it suffices to show $\m \in \V_{(\g,\even{\g})}(M)$, i.e., that every element $\zeta \in \H^\bullet(\g,\even{\g};\CC)$ which annihilates $\Ext_{(\g,\even{\g})}^\bullet(M,M)$ is an element of $\m$. Consider such a $\zeta$ and an extension $0 \to M \to \ldots \to M \to 0$. We wish to show $\resmap \zeta$ annihilates $\Ext_{(\g,\a)}^\bullet(M,M)$ -- it is only clear that $\resmap \zeta$ annihilates $\resmap \left( \Ext_{(\g,\even{\g})}^\bullet (M,M)\right)$, i.e., it may not annihilate some sequences which split on restriction to $\a$ but not on restriction to $\even{\g}$.

  Coversely, consider an element $\m \in \V_{(\g,\even{\g})}(M) \cap \Phi\left(\V_{(\g,\a)}(\CC)\right)$. Choose $\widetilde{\m} \trianglelefteq \H^\bullet(\g,\a;\CC)$ such that $\m = \resmap^{-1}(\widetilde{\m})$. We wish to show $ \widetilde{\m} \supseteq \Ann_{\H^\bullet(\g,\a;\CC)}\Ext^\bullet_{(\g,\a)}(M,M)$. Choose $\zeta \in \Ann_{\H^\bullet(\g,\a;\CC)}\Ext_{(\g,\a)}^\bullet(M,M)$. Because the restriction homomorphism is surjective (on the level of rings), there exists an element $\hat \zeta \in \H^\bullet(\g,\even{\g};\CC)$ which restricts to $\zeta$. Such a $\hat \zeta$ annihilates all elements of $\Ext_{(\g,\even{\g})}^\bullet(M,M)$, since the action is defined by tensor product (over $\CC$) of exact sequences. Thus $\hat \zeta \in \Ann_{\H^\bullet(\g,\even{\g};\CC)}\Ext_{(\g,\even{\g})}^\bullet(M,M) \subseteq \m$, and by construction $\resmap \hat \zeta = \zeta \in \widetilde{\m}$.
\end{proof}
% END DELETED PART


\begin{corollary}
  Suppose $\V_{(\g,\a)}(\CC) \to \V_{(\g,\even{\g})}(\CC)$ is a closed embedding. If a variety $X \subseteq \V_{(\g,\even{\g})}(\CC)$ is realized by a  $(\g,\even{\g})$-module, then $X \cap \V_{(\g,\a)}(\CC)$ is realized by a $(\g,\a)$-module.
\end{corollary}

\section{Connectedness of support varieties}
\label{sec:connectivity}

This section investigates connectedness of support varieties, motivated by Benson's presentation \cite{MR1634407}.

\begin{proposition}
  Let $\g = \even{\g} \oplus \odd{\g}$ be a classical, stable, and polar Lie superalgebra with $\a \leq \even{\g}$ a natural subalgebra. Suppose $\Phi(\V_{(\g,\a)}(M)) = X \cup Y$ with $X \cap Y = \{0\}$. Then there exist modules $M_1$ and $M_2$ such that $M = M_1 \oplus M_2$, $X = \Phi(\V_{(\g,\a)}(M_1))$, $Y = \Phi(\V_{(\g,\a)}(M_2))$, and \[
    \Phi(\V_{(\g,\a)}(M)) = \Phi(\V_{(\g,\a)}(M_1)) \cup \Phi(\V_{(\g,\a)}(M_2)).
  \]
\end{proposition}
\begin{proof}
  By realizability for $(\g,\even{\g})$, because $\Phi(\V_{(\g,\a)}(M))$ is a closed conical subvariety of $\V_{(\g,\even{\g})}(M)$, there exist $M_1$ and $M_2$ such that $\Phi(\V_{(\g,\a)}(M)) = \V_{(\g,\even{\g})}(M_1) \cup \V_{(\g,\a)}(M_2)$. Using this fact, we may compute:
  \begin{align*}
    \Phi(\V_{(\g,\a)}(M)) &= \V_{(\g,\even{\g})}(M_1) \cup \V_{(\g,\even{\g})}(M_2) \\
                         &= \left(\Phi(\V_{(\g,\a)}(\CC)) \cap \V_{(\g,\even{\g})}(M_1)\right) \cup \left(\Phi(\V_{(\g,\a)}(\CC)) \cap \V_{(\g,\even{\g})}(M_2)\right) \\
    &= \Phi(\V_{(\g,\a)}(M_1)) \cup \Phi(\V_{(\g,\a)}(M_2)).
  \end{align*}
\end{proof}


% \begin{proposition}
%   Let $\g = \even{\g} \oplus \odd{\g}$ be a classical Lie superalgebra, $\a \leq \even{\g}$ an even subalgebra, and $M$ a finite-dimensional $(\g,\a)$-module. If $\V_{(\g,\a)}(M) = \V_1 \cup \V_2$ with $\V_1 \cap \V_2 = \{0\}$, then $M \cong M_1 \oplus M_2$ with $\V_i = \V_{(\g,\a)}(M_i)$.
% \end{proposition}
% \begin{proof}
%   The proof of Benson's 5.12.1 carries over verbatim with the exception of requiring a tensor product theorem and the fact $\V_{(\g,\a)}(L_\zeta) = \V_{(\g,\a)}\left<\zeta\right>$. The first will hold for $\gl(m|n)$, and more generally for any superalgebra which satisfies the tensor product theorem for $(\g,\even{\g})$, whereas the second will hold once I work out the highlighted realizability kink above.
% \end{proof}


%%% Local Variables:
%%% TeX-master: "../dissertation.tex"
%%% End:


\bibliographystyle{alpha}{}
\bibliography{references}

\appendices

%
\chapter{Supersymmetry Theory}
\label{ap:SUSY}

Here we provide a brief sketch of supersymmetry theory. More details may be found in the book ``Supersymmetry for Mathematician'' \cite{varadarajan}

%%% Local Variables:
%%% TeX-master: "../dissertation.tex"
%%% End:
%
\chapter{Relative Cohomology of Hochschild}
\label{ap:rel-coh}

%%% Local Variables:
%%% TeX-master: "../dissertation.tex"
%%% End:

\end{document}

